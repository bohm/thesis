\section{The bin stretching problem}\label{sec:binstretchdef}

Let us return to our semi-online container repacking problem from
Section~\ref{sec:example}, which we now define formally:

\begin{prb}[\binstretch]~

\noindent\textbf{Input:}

\begin{itemize}
\item An integer $m$ --  the number of allowed bins. All bins are initially empty.
\item A sequence of items $I=i_1, i_2, \ldots$ given online one by one. Each item has
a {\it size} $s(i) \in [0,1]$ and must be packed immediately and irrevocably into one of the $m$ bins; the next item
is revealed only after the previous one is packed.
\end{itemize}

\noindent\textbf{Parameter:} The {\it stretching factor} $R\geq 1$.

\noindent\textbf{Output:} Partitioning (packing) of $I$ into bins $B_1,\ldots,B_m$
so that $\sum_{i\in B_j}s(i)\leq R$ for all $j=1,\ldots,m$.

\noindent\textbf{Guarantee:} there exists a packing of all items in $I$ into $m$
bins of capacity $1$. 

\noindent\textbf{Goal:} Design an online algorithm with the stretching factor $R$
as small as possible which packs all input sequences satisfying the
guarantee.
\end{prb}
\medskip

The aforementioned stretching factor is just another name for the
competitive ratio of our algorithm in the standard terminology of
online algorithms (see Definition~\ref{dfn:compratio}). Still, when
designing algorithms for \binstretch it makes more sense to think
about it as a parameter, as a maximum capacity that we want to
maintain in all bins.

An attentive reader might object to the fact that we have defined
\binstretch as a semi-online problem, as there is no tangible advice
given to an input instance -- the only thing we know is that some
optimal packing exists.

The advice given is hidden in the fact that we actually only consider
scaled instances where the unknown optimal packing is into $m$ bins of
capacity $1$. If we go back to our example with repacking containers,
we can see that the advice is actually the original load of the
containers (e.g. that every container was at most $75\%$ full).

As we can see from the nomenclature above -- bins, load, packing -- we
will think of the problem as if it is related to \binpacking
(Problem~\ref{prb:binpacking}). This is also how the literature
discusses the problem.  It is however important to keep in mind that
the problem is not a variant of \binpacking -- the crucial difference
is that we are not allowed to open as many bins as we want, which is
legal in \binpacking.

In fact, \binstretch is more closely related to the problem \scheduling
(Problem~\ref{prb:scheduling}). In the scheduling terminology, we
would call it \textsc{Online Scheduling with Known Optimal Makespan},
and we would describe it this way:
\goodbreak

\begin{prb}[\binstretch / \textsc{Online Scheduling with Known Optimal Makespan}]~
\label{prb:binstretchscheduling}
\smallskip

\noindent\textbf{Input:}

\begin{itemize}
\item An integer $m$ -- a number of identical (equally fast) \emph{machines};
\item A sequence of \emph{jobs} $I=i_1, i_2, \ldots$ given online one by one. Each job has
a \textit{processing time} $s(i) \in [0,1]$ and must be assigned immediately and irrevocably
to one of the machines. The jobs cannot be paused or reassigned later.
\end{itemize}

\noindent
\textbf{Output:} A \emph{schedule} assigning all input items to $m$ machines, with the
load of the busiest machine (called \emph{makespan}) being some value
$R$.

\noindent
\textbf{Guarantee:} There exists a schedule which assigns all jobs so that
the makespan is at most $1$.

\noindent
\textbf{Goal:} Design an online algorithm minimizing the makespan $R$.
\smallskip
\end{prb}

\paragraph{A pinch of notation.} For a bin $B$, we define the
\textit{load of the bin} $s(B) = \sum_{i \in B} s(i)$. Unlike
$s(i)$, $s(B)$ can change during the course of the algorithm, as we
pack more and more items into the bin. To easily differentiate between
items, bins and lists of bins, we use lowercase letters for items
($i$, $b$, $x$), uppercase letters for bins and other sets of items
($A$, $B$, $X$), and calligraphic letters for lists of bins ($\calA$,
$\calC$, $\calL$).

\paragraph{Input scaling.} We have defined \binstretch as a problem
where all input items are of size between $0$ and $1$; in other words,
all sizes are set up so that the guaranteed optimum packing is $1$.
This is natural for a generic definition, but writing and especially
adding fractions such as $\frac{1}{2}$ and $\frac{7}{16}$ becomes
tiresome after a while.

Therefore, in all three sections of the thesis, we rescale the
instance so that the capacity of the bins in the optimum packing
becomes some positive integer $S \ge 1$ and the capacity of a bin for
the algorithm will also be some positive integer $R \ge S$. The
resulting stretching factor will therefore be $R/S$.

It is worthwhile to keep in mind that our rescaling still allows
fractional values; while bins have integral capacity both for the
algorithm and for the optimal guarantee, the input item size $s(i)$
can be any value in the interval $(0,S]$.

\section{Bin Stretching as a game}\label{sec:binstretchgame}

A basic and essential feature of any online problem -- with
\binstretch being no exception -- is that any online algorithm is
actually an algorithm for strategy in a two-player game.

What do we mean by that? If we recall the definition of the
competitive ratio, we see that we measure our algorithm on the actual
worst case that can happen, and this worst case is actually
\emph{adapting} to what the algorithm does for the preceeding items.

Therefore, we can see the creator of the worst case as a person who
plays a game with our algorithm, waiting for its move (in our case,
where it packs the previous item) and then sending the next item so
that the algorithm's decision is the worst possible.

In the online algorithm literature, we call this player
\adversary. The other player, naturally, is called \algo.

\begin{dfn} 
A \emph{combinatorial two-player game} is a game with two
players with perfect information (the game state is fully revealed to
both players). Each player has a set of moves (which does not need to
be symmetric) and they alternate in selecting one move each.

A set of game states are denoted as winning the game for the first
player and another set as winning for the second player. The two sets
are known to both players in advance.

A player \emph{wins} the game if it reaches its own winning state in a
finite set of moves. A player has a \emph{winning strategy} if
it can reach a winning state regardless of the moves of the other
player.
\end{dfn}

% \todo{Find reference for the earliest mention of this.}

\begin{thm}
Fix a target competitive ratio $c$. An online problem along
with $c$ corresponds to a combinatorial two-player game between
players \algo and \adversary. The player \algo has a winning
strategy if and only if there exists a $c$-competitive algorithm for
the online problem. The player \adversary has a winning strategy
if and only if no $c$-competitive algorithm exists.
\end{thm}

One can argue (in jest) that the main consequence of this theorem is
that the literature on online algorithms loves to use terminology from
combinatorial game theory; it is the author's personal experience that
imagining an adversary and constantly thinking about its moves is
useful when designing a new online algorithm.

However, the more important consequence for us is that if we can
evaluate the entire decision tree for a game that corresponds to an
online problem with a fixed ratio $c$, we can learn whether a
$c$-competitive algorithm exists or not. There is a classical
algorithm for evaluating a game tree called \emph{the minimax
algorithm} which is effective as long as the tree can be somehow
bounded.

So far, everything we stated about game theory applies in full
generality, but we will focus on \binstretch exclusively; therefore,
we restate the game that corresponds to \binstretch:

\begin{dfn}
Fix a number of bins $B$ and a stretching factor $S$. The \emph{bin stretching game}
is defined as follows:

\begin{enumerate}
\item It is a two-player game between \algo and \adversary, and the player \adversary starts.
\item The player \adversary moves by sending any item $i \in (0,1]$ with the restriction
that $i$ along with all previous items can be packed into $B$ bins of capacity $1$.
\item The player \algo moves by selecting a bin where the item is packed.
\end{enumerate}

The winning states are defined thus:
\begin{itemize}
\item A state is winning for \algo if all bins are packed strictly below $S$ and \adversary
has cannot make any further moves.
\item A state is winning for \adversary if one bin has total load at least $S$.
\end{itemize}
\end{dfn}

As you can see, we have formulated the bin stretching game with the
player \algo trying to pack strictly below $S$. This way, a winning
strategy for the player \adversary immediately implies that no online
algorithm for \binstretch with stretching factor less than $S$ exists.

The two main obstacles to implementing a search of the described two player
game are the following:

\begin{enumerate}
\item \adversary can send an item of arbitrary small size;
\item \adversary needs to make sure that at any time of the game, an offline
  optimum can pack the items arrived so far into three bins of capacity $1$.
\end{enumerate}

We discuss our solutions as well as the implementation details in Chapter~\ref{chap:lb}.