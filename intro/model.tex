\section{The online model of computation}

Before we continue with the investigation of the problem of repacking
containers, we specify the model of computation that we are
considering.

For those not in the know of the theoretical computer science
terminology, an \textit{online algorithm} conjures an image of a
program which is hosted on a server on the internet. This may indeed
be so, but the definition actually predates the internet and goes to
the 1970s, where the word ``online'' implied something arriving
literally ``on a line'', such as on a conveyor belt.

In the online model, the input to the algorithm is not known in
advance, but it arrives sequentially, one by one -- exactly like
luggage on the conveyor belt. After each part of the input arrives,
the algorithm must \textit{immediately} and \textit{irrevocably}
decide what to do with this particular piece of the input. Most
commonly, the decision is where to send a packet, which processor core
should be selected for the current task, or which container should be
used to pack the incoming item. As mentioned, after a decision is
made, it cannot be reverted. 

Only after a decision is made, the next item in the sequence is
revealed to the algorithm, and the same decision process is repeated;
of course even if two items arrive which are exactly the same, the
algorithm can decide differently on them, since it knows the history
of the instance.

After the entire input is processed, we look at the whole output of
the algorithm and consider it a \emph{solution} of the online problem.
Naturally, the resulting solution is also a valid solution for the
original non-online optimization problem, but usually is worse than
what a normal polynomial-time algorithm for that optimization problem
would produce.

\begin{dfn}
An \emph{online problem} is an optimization problem where some part of
the input instance arrives sequentially, one by one. After each part
arrives, it must be processed immediately and irrevocably, and the
next part of the input instance arrives only after the processing of
the previous part is complete.

An \emph{online algorithm} is an algorithm that outputs some solution
of an online problem. Given an instance $I$ of the online problem, we
can measure the performance of the online algorithm by its output's
\emph{value} on the function being optimized. We denote its value on
instance $I$ as $\ALG(I)$.
\end{dfn}

As examples of online problems, we can list the two that are arguably
the most well-studied: scheduling and bin packing.

\begin{prb}[\scheduling]~
\smallskip

\noindent\textbf{Input:}
\begin{itemize}
\item A number $m$ -- the number of available equal-speed machines.
\item A sequence $j_1, j_2, …, j_n$ of jobs, arriving online (one by one). Each job $j$ has its own non-negative \emph{processing time} $p(j)$. The processing time is revealed when the job arrives. The number of jobs $n$ is not known in advance.
\item Once a job appears, the algorithm must assign the job to one machine; the next job is revealed after the decision is made.
\end{itemize}

\noindent\textbf{Output:} A schedule (assignment) of jobs to machines,
along with the makespan, which is the sum of the processing times on
the busiest machine.

\noindent\textbf{Goal:} Design an online algorithm that minimizes the
makespan.
\end{prb}

\begin{prb}[\binpacking]~
\smallskip

\noindent\textbf{Input:}
\begin{itemize}
\item A sequence $i_1, i_2, …, i_n$ of items, arriving online (one by one). Each item $i$ has its own non-negative \emph{size} $s(i), s(i) \le 1$. The size is revealed when the item arrives.
\item Once an item appears, it must be assigned to a \emph{bin} such that the total load of the bin after the assignment is at most $1$. The algorithm can open a new bin (with load $0$) at any time, and can open any number of bins.
\end{itemize}

\noindent\textbf{Output:} A number of bins used by the algorithm $B$, as well as
the assignment of all input items into these bins.

\noindent\textbf{Goal:} Design an online algorithm that minimizes the number of bins $B$.
\end{prb}

\medskip

The goal of algorithm designers is not only designing algorithms that
perform well in practice, but also those that \emph{verifiably}
perform well. Since online algorithms are designed only for
optimization problems, one could suggest to compute the value of the
best possible online algorithm for the problem and compare our
algorithm to that value. This is actually quite hard, since in almost
all cases there is no good way to compute the value of the optimal
online algorithm.

Instead, we compare our algorithm's performance to an optimum that
actually is not online at all, but has access to the whole input at
the same time. We define it as follows: 

\begin{dfn}\label{dfn:compratio}
An \emph{offline optimal algorithm} is an algorithm that optimally
solves the online problem when given the entirety of the instance at the start
of the program. On an instance $I$, we denote its value as $\OPT(I)$.

For a minimization problem, we say an online algorithm has
\emph{competitive ratio $c$}, if for all instances $I$,

\[ \ALG(I) \le c \cdot \OPT(I). \]

For a maximization problem, we say an online algorithm has
\emph{competitive ratio $c$} if for all $I$,

\[ \OPT(I) \le c \cdot \ALG(I). \]

\end{dfn}

It might seem confusing to some readers that the competitive ratio is
differently defined for minimization and maximization problems; this
is done so chiefly to make sure that the competitive ratio $c$ is
always in the interval $[1,\infty]$ of real numbers.

So far we have not spoken at all about the running time of online
algorithms. This was intentional: there are no time or space
restrictions on their computation. All we need is that they are
algorithms, i.e., they finish in finite time. Despite this relaxed
view of algorithmic runtime, almost all online algorithms turn out to
run in polynomial time, many even in linear time.

% \todo{Mention that the computational complexity is unlimited. Also check
% the definitions.}

\section{Knowing a little in advance: the semi-online model}

As we discussed above, an online algorithm cannot work with any assumptions
about the input or its order -- which means that if an online algorithm works
excellently on larger input instances but poorly on a sequence of short length,
its performance will be judged exactly on those short sequences.

One can easily see that ``assuming nothing'' is both a great advantage
and at the same time a big disadvantage of the online computation
model. Consider for instance the internet -- while our algorithms
should behave well under any scenario, it is not true that the very
worst scenario will always come true. Indeed, many routing and
scheduling algorithms make use of previously computed statistics and
assumptions about the data to behave well only in an average case or
even just as a \textit{heuristic}, i.e. without any provable guarantees.

If we wish to make our rule of ``assuming nothing'' weaker, we have
several different options, all already studied in literature:

\begin{enumerate}

\item We could assume that the incoming items are not random, but
are coming from a probabilistic distribution whose parameters we know.
This is called \textit{stochastic online optimization}.

\item We could assume that we are given some advice from an entity
that has knows much more about the input than we do (a likely
candidate for such an oracle would be NSA). While this entity can give
us any advice at all, the information must be relatively small, for
the model to make sense. After our algorithm receives this arbitrary
but limited advice, it must make use of it to solve the online problem
in an effective way. This model is called \textit{online optimization
with advice}.

\item There is a common sense weakening of the \textit{optimization
with advice} model from the previous point, and we can describe it
with an example. Suppose that our online algorithm is packing a
sequence of apples into crates as best as it can. If we assume any
advice at all, the algorithm could receive some information about the
average curvature of the current batch of apples. The curvature may
be potentially useful for packing, but clearly in real life there is no
way an apple orchard can provide such an oddly specific piece of data.

A much more reasonable request would be to provide the total number of
apples, the total weight of the apples, or the estimated quality of
the current season's fruit. In other words, in this model, we get a
reasonable and pre-defined piece of information about the instance in
advance, but our algorithms still need to solve the instance in an
online way.

This model is called \textit{semi-online optimization}, and it is the
one we focus on in this thesis. We mention the other two models briefly
in Section~X. \todo{Mention related work section.}

\end{enumerate}