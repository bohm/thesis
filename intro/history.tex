\section{History}\label{sec:1:history}

%\todo{History from journal papers is below.}

\subsection{History of Bin Stretching}

\binstretch (or \textsc{Online scheduling on identical machines with
known makespan}, if you prefer the scheduling terminology) is a
natural extension of the \textsc{Online scheduling} problem, but it
was not investigated until the 1990s.

The first result on \binstretch was a lower bound of $4/3$ for two
bins and a matching algorithm, which was discovered by Kellerer,
Kotov, Speranza and Tuza in 1997~\cite{KeKoST97}, in a paper focused
primarily on semi-online partitioning problems.

The name \binstretch has been proposed by Azar and Regev in
1998~\cite{azar98,azar01}, in the first paper dedicated to this
problem. They extended the lower bound of $4/3$ to any number of bins
and gave an online algorithm with a stretching factor $1.625$.

\paragraph{The first Bin Stretching algorithm.} Let us dive a little
bit more into the algorithms proposed by Azar and Regev, so that we
can see how the ideas evolved in subsequent work. Imagine that we are designing
an algorithm with stretching factor $1+\alpha$, with $\alpha$ being the extra space.
We can notice that whether a bin is loaded at most $\alpha$ is an important threshold;
one reason might be because a bin of load at most $\alpha$ can still accept an item of size $1$.

Using that threshold, Azar and Regev design two algorithms:

\begin{algorithm}
\caption{}
% \label{alg:bounded}
\begin{algorithmic}[1]
\For{the next incoming item $i$}
\State \algorithmicif\ there is a non-empty bin which stays below $\alpha$ with $i$, pack it there.
\State \algorithmicif\ there is a bin which is already above $\alpha$ and $i$ fits, pack it there.
\State \algorithmicif\ there is an empty bin where $i$ fits below $\alpha$, pack it there.
\State Finally, try packing it into the least-loaded machine where $i$ fits.
\EndFor 
\end{algorithmic}
\end{algorithm}

\begin{algorithm}
\caption{}
% \label{alg:bounded}
\begin{algorithmic}[1]
\For{the next incoming item $i$}
\State \algorithmicif\ there is \emph{any} bin which stays below $\alpha$ with $i$, pack it there.
\State \algorithmicif\ there is bin which is already above $\alpha$ and $i$ fits, pack it there.
\State Finally, try packing it into the least-loaded machine where $i$ fits.
\EndFor 
\end{algorithmic}
\end{algorithm}

Both of these algorithms actually achieve a stretching factor of $5/3$
(in other words, they never fail when $\alpha \ge 2/3$). A smart
combination of the two leads to an algorithm with stretching factor
$1.625$.

\todo{Fix section references.}

The next algorithmic progress came as a consequence of the work of
Cheng, Kellerer and Kotov in 2003~\cite{cheng2003}. They design an
online algorithm with a competitive ratio of 1.6 for a more general
problem that also encompasses \binstretch (see Section~X for the
definition of the problem). The same algorithm therefore achieves a
stretching factor of 1.6 for \binstretch.

\paragraph{Classification and bunching techniques.} The next decrease
of the upper bound on the stretching factor appeared ten years later
in a result of Kellerer and Kotov \cite{kellerer2013}. Their algorithm
achieves a stretching factor of $11/7 \approx 1.571$ using two ideas
that appear in subsequent work as well as our thesis; we focus on them
now.

% \todo{Make a nicer transition between the paragraphs.}

The first idea is to \emph{classify} items into groups based on their
size, which in \cite{kellerer2013} are selected as follows:

\begin{center}
  \begin{tabular}{ l | c | c | r }
    Class: & Small items & Medium items   & Large items \\ \hline
    Sizes: & $(0,4/7]$   & $(4/7, 11/14]$ & $(11/14,1]$ \\ 
  \end{tabular}
\end{center}

Just looking at the sizes alone, we can infer some properties that may
be useful: for instance, the upper bound $4/7$ for small items is
precisely the value of $\alpha$ for an algorithm with stretching
factor $11/7$. As for the medium items, we can see that a pair of them
always fits together -- in fact, the upper bound for medium items is
the largest value for which it is true. Plus, when we pack two of them
together, we get a load of at least $8/7$, which is allowed for the
algorithm while these items need to be in separate bins in the optimal
packing \footnote{This is true for any size above $1/2$, so there must be
another reason for selecting the lower bound to be $4/7$.}.

Similarly to item classes, Kellerer and Kotov also classify bins based
on their current load, including one more class that is defined in a
different way:

\begin{center}
  \begin{tabular}{ r | c | c | c | c | l }
    Class: & Tiny bins & Small bins & Medium bins & Large bins & Huge bins \\ \hline
    Load:  & $(0,2/7]$ & $(2/7, 4/7]$ & $(4/7, 11/14]$ & $(11/14,1]$ & $(1,11/7]$ \\ 
  \end{tabular}
\end{center}

\begin{center}
  \begin{tabular}{ r | l }
    Class: & Large-item bins \\ \hline
    Condition: & Contains a large item. \\ 
  \end{tabular}
\end{center}


The algorithm works in two phases. The algorithm for the \emph{first phase} is the following
one:

\begin{algorithm}
\caption{First phase of Kellerer and Kotov}
% \label{alg:bounded}
\begin{algorithmic}[1]
\For{the next incoming item $i$}
\If{$i$ is small}
\State \algorithmicif\ $i$ fits into a large-item bin, pack it there.
\State \algorithmicif\ $i$ fits into a small bin and the bin remains small, pack it there.
\State Otherwise, put $i$ into an empty bin.
\EndIf
\If{$i$ is medium}
\State If $i$ fits into a medium bin, pack it there.
\State Otherwise, pack $i$ into an empty bin.
\EndIf
\If{$i$ is large}
\State Pack $i$ into a small bin with the largest load.
\State \algorithmicif\ none exist, pack $i$ into an empty bin.
\EndIf
\If{the ratio of medium $\&$ small bins to empty bins is more than $3:1$}
\State End the first phase.
\EndIf
\EndFor 
\end{algorithmic}
\end{algorithm}

We have so far employed just the bin classification; only in the
\emph{second phase} comes the bunching technique into play. Since the
first phase ends when the ratio of small and medium bins to non-empty
bins is higher than $3:1$, the algorithm creates groups (bunches) of 4
bins and packs incoming items one group at a time; the next group is
used only when it is impossible to pack into the previous one.

Roughly said, this 3:1 bunching is used to show the following: either
a fourth big item arrives for this bunch (and fits into the one empty
bin) or there were more than $4$ units of items put in all four bins
together and the algorithm must finish correctly.

\todo{Explain the next algorithmic improvement of Gabay, Brauner and
Kotov.}

\paragraph{Computer lower bounds.} Interestingly, the setting with a
small fixed number of bins allows better lower bounds on the
stretching factor. Gabay, Brauner and Kotov~\cite{gabay2013lbv2} give
a new lower bound of $19/14$ for the setting where there are exactly
three bins, i.e. $m=3$. \footnote{Subsequently to our work, the
preprint \cite{gabay2013lbv2} was updated to include the lower bound
of $19/14$ for $m=4$ as well~\cite{gabay2013lbv3}.}

They create the lower bound instance using a computer search based on
the ideas which we have described in Section~X and which we will see
in full detail in Chapter~4, where we also show our extensions to
their approach.

The lower bounds for $m=3$ cannot be easily translated into a lower
bound for a larger $m$; for example, if we modify the instance by
adding new bins and a corresponding number of items of size 1 (that
must use exactly the new bins in the optimum), the semi-online
algorithm still could use the additional capacity of $\alpha$ in the
new bins to its advantage.

\section{Related topics}\label{sec:1:related}

\subsection{Bin packing}\label{sec:1:binpackinghistory}

The NP-hard offline problem \binpacking was originally proposed by
Ullman~\cite{ullman71} and Johnson~\cite{johnson73} in the
1970s. Since then it has seen major interest and progress, see the
survey of Coffman et al.~\cite{coffman13} for many results on
classical \textsc{Bin Packing} and its variants. 

\subsection{Online scheduling}\label{sec:1:schedulinghistory}

Alongside \binpacking, \scheduling is probably the other most
well-known problem for the online computational model. The first
algorithm for \scheduling was given by Graham~\cite{Graham66},
achieving a competitive ratio of $2-\frac{1}{m}$ for $m$ machines.

This ratio was later improved to a constant factor independent on $m$;
the currently best online algorithm for \scheduling is
$1.9201$-competitive \cite{}, and there is a lower bound showing that
no online algorithm for \scheduling can be better than
$1.88$-competitive \cite{}.

Since \scheduling clearly belongs to the greater research area of
scheduling theory, there are incredibly many variants that study
different performance metrics, special job sizes, machines with
varying properties and much more. See the survey of Albers
\cite{alberssurvey} on \scheduling for many more results and open
problems in the area.

\subsection{Semi-online scheduling}

As noted, \binstretch can be formulated as online scheduling on
$m$ identical machines with known optimal makespan. Such algorithms
were studied and are important in designing constant-competitive
algorithms without the additional knowledge, e.g., for scheduling in
the more general model of uniformly related
machines~\cite{AsAFPW97,BeChKa00,EbJaSg09}.

For scheduling, also other types of semi-online algorithms are
studied. Historically first is the study of ordered sequences with
non-decreasing processing times~\cite{Graham69}. Most closely related
is the variant with known sum of all processing times studied
in~\cite{KeKoST97}. If we know the optimal makespan, we can always pad
the instance by small items at the end so that the optimal makespan
remains the same and the sum of processing time equals $m$ times the
optimal makespan. Thus one could expect that these two quantities are
interchangeable. However, when the sum of all processing times is
known, the currently best results are a lower bound of $1.585$ and an
algorithm with ratio $1.6$, both
from~\cite{DBLP:journals/tcs/AlbersH12}. This shows, somewhat
surprisingly, that knowing the actual optimum gives a significantly
bigger advantage to the semi-online algorithm over knowing just the
sum of the processing times. See Pruhs et al.~\cite{PST04} for a
survey of other results on (semi-)online scheduling.

\todo{Add more semi-online results, stochastic optimization and advice optimization.}

% \begin{algorithm}
% \caption{Algorithm $\chase_{d,r}$ for $r$-bounded instances with $v_0 = 0$}
% \label{alg:bounded}
% \begin{algorithmic}[1]
%   \If {$\hp_k \cap F_i = \emptyset$ for all $k \in [d]$}
%   \State Let $B(s,r)$ be the minimum-volume ball containing $F_i$
%   \State Update $B_i \leftarrow B(s,r)$ and $v_i \leftarrow s$
%   \State Recenter so that $s = 0$
%   \State Let $v_k(i)$ be location of $\chase_{d-1}(0,r,\hp_k)$ for each $k \in [d]$
%   \Else
%   \State Let $k$ be smallest index such that $\hp_k \cap F_i \neq \emptyset$
%   \State Update $B_i \leftarrow B_{i-1}$ and $v_i \leftarrow v_k(i)$
%   \EndIf  
% \end{algorithmic}
% \end{algorithm}

