\section{Verifying the offline optimum guarantee}\label{subsec:test}

When we evaluate a turn of the \adversary, we need to create the list
$L = \{0,1,\ldots,y\} \subseteq \{0,1,\ldots,T\}$ of items that
\adversary can actually send while satifying the \binstretch
guarantee. We employ the following steps:

\begin{algorithm}
\caption{Procedure $\MaxFeas$}
\noindent Input is a bin configuration $C = (\calL,\calI)$ with $n$ items and $m$ bins.
\begin{algorithmic}[1]
\State Calculate an upper bound $UB$ on the value $y$.
\State Calculate a lower bound $LB$ on the value $y$ using an online best fit algorithm (Section~\ref{sec:4:dynprogbounds}).
\If{$LB < UB$}
\State Do a linear search on the interval $\{UB,UB-1,\ldots,LB\}$
using a procedure $\Query$($\calI \cup \{i\})$ that queries the cache
whether $\calI \cup \{i\}$ is feasible or not (Section~\ref{sec:4:caching}).
\State Update $LB, UB$ to be the best values confirmed feasible/infeasible by $\Query$.
\EndIf
\If{$LB < UB$}
\State Update $LB$ using the (offline) algorithm \bfd.
\EndIf
\If{$LB < UB$}
\State Compute the exact value of $y$ using a dynamic program $\DynprogMax$ (Section~\ref{sec:4:dynprogmax}).
\EndIf
\State \Return $y = LB = UB$.
\end{algorithmic}
\end{algorithm}

\subsection{Upper and lower bounds}\label{sec:4:dynprogbounds}

If we wish to compute the value $y$ directly, we need to call the
dynamic program $\DynprogMax$, whose complexity is $O(n \cdot S^m)$ in
the worst case (and that is if we ignore potential slowdowns via
hashing).

This is polynomial when $m$ is a constant, but already for $m=3$ and
especially when $4 \le m \le 9$ such a call per every game state
becomes prohibitively expensive. Therefore we employ several tricks
that help us avoid computing $\DynprogMax$ in some game states.

First, we compute a lower and upper bound on the maximum feasible item
size $y$; if they match, we have found the correct value of $y$.  The
upper bound will be $UB \defeq \min(y', m\cdot S - t)$, where $y'$ is
the maximum feasible value that was computed in the previous vertex of
\adversary's turn, and $t$ is the total size of all items in the
instance. The second term is therefore the sum of all items that can
arrive in this instance.

\paragraph{Online Best Fit.} To find the first lower bound on $y$ quickly, we
employ an online bin packing algorithm \obf. This algorithm maintains
a packing of items $\calI$ to $m$ bins of size $S$ during the
evaluation of the algorithm $\Sequential$, packing each item as it is
selected by the player \adversary. The algorithm \obf packs each item
$i$ into the most-loaded bin where the item fits.

Once the algorithm $\Sequential$ selects a different item $i'$ and
evaluates a different branch of the game tree, the online algorithm
removes $i$ from its bin and inserts $i'$ to the most-loaded bin
where $i'$ fits.

As \obf maintains just one packing, which may not be optimal, it can
happen that \obf is unable to pack the next item $i$ even though $i$
is a feasible item. In that case, we mark the packing as inconsistent
and do not use the lower bound from \obf until its online packing
becomes feasible again.

If the packing maintained by \obf is still feasible, we return as the
lower bound value $LB$ the amount of unused space on the least-loaded
bin.

The main advantage of \obf is that it takes at most $O(m)$ time per
each step, and especially for the earlier stages of the evaluation its
returned value can match the value of $y$.

\paragraph{Checking the cache.} Next, if a gap still remains between
$LB$ and $UB$, we try to tighten it by calling a procedure \Query
which queries the cache of feasible and infeasible item multisets.
The procedure has a ternary answer -- either an item multiset $\calI
\cup \{j\}$ was previously computed to be feasible, or it was computed
to be infeasible, or this item set is not present in the cache at all.

We update $LB$ to be the largest value which is confirmed to be
feasible, and update $UB$ to be $1$ less than the smallest value
confirmed to be infeasible.

\paragraph{Best Fit Decreasing.} If the values $LB$ and $UB$ are still
unequal, we employ a standard offline bin packing algorithm called
\bfd. \bfd takes items from $\calI$ and first sorts them in decreasing
order of their sizes. After that it considers each item one by one in
this order, packing it into a bin where it ``fits best'' -- where it
minimizes the empty space of a bin. We can also interpret it as first
sorting the items in decreasing order and then applying the algorithm
\obf defined above.

As for its complexity, \bfd takes in the worst case $O(m \cdot \calI)$
time. It does not need to sort items in $\calI$, as the internal
representation of $\calI$ keeps the items sorted.

As with \obf, the lower bound $LB$ will updated to the maximum empty
space over all $m$ bins, after \bfd has ended packing. Such an item
can always be sent without invalidating the \binstretch guarantee.

\subsection{Procedure $\DynprogMax$}\label{sec:4:dynprogmax}

Procedure \DynprogMax is a sparse modification of the standard dynamic
programming algorithm for \textsc{Knapsack}. Given a multiset $\calI,
|\calI| = n$ on input, our task is to find the largest item $y$
which can be packed together with $\calI$ into $m$ bins (knapsacks) of
capacity $S$ each.

% Note that using this formulation it can happen that $|y| = 0$ or even
% that $\calI$ itself is infeasible; however, neither case can happen in
% our actual use of the procedure.

We use a queue-based algorithm that generates a queue $Q_i$ of all
valid $m$-tuples $(a,b,c,\ldots)$ that can arise by packing the first
$i$ items. We do not need to remember where the items are packed, only
the loads of the bins represented by the $m$-tuple.

To generate a queue $Q_{i+1}$, we initialize it to be an empty queue.
Next, we traverse the old queue $Q_i$ and add the new item
$\calI[i+1]$ to all bins as long as it fits, creating up to $m$ new
tuples that need to be added to $Q_{i+1}$.

Unsurprisingly, we wish to make sure that we do not add the same tuple
several times during one step. We can use an auxiliary $\{0,1\}$ array
for this purpose, but we have ultimately settled on a hash-based
approach.

We use a small array $A$ of $64$-bit integers (of approximately $2^{10} -
2^{13}$ elements). When considering a tuple $t'$ that arises from
adding $i$ to one of the bins in the tuple $t$, we first compute the
hash $h(t')$ of the tuple $t'$. Since we use Zobrist hashing (see
Section~\ref{sec:4:caching}), this operation takes only constant time.

Next, we consider adding $t'$ to the queue $Q_{i+1}$. We use the first
$10-13$ bits of $h(t')$ (let $f$ denote their value) and add $t'$ to
$Q_{i+1}$ when $A[f] \neq h(t')$ -- in other words, when the small
array $A$ contains something other than the hash of $t'$ at the
position $f$. We update $A[f]$ to contain $h(t')$ and continue.

While our hashing technique clearly can lead to duplicate entries in
the queue, note that this does not hurt the correctness of our
algorithm, only its running time in the worst case.

We continue adding new items to the tuples until we do $n$ steps and
all items are packed. In the final pass of the queue, we look at the
empty space $e$ in the least-loaded bin. The output of $\DynprogMax$
and the value of $y$ is the maximum value of $e$ over all tuples in
the final pass.

Ignoring the collisions of the hashing scheme (which can happen but
will not play a big role if we compute the expected running time based
on our randomized hashing function), the time complexity of the
procedure \MaxFeas is quite high in the worst case: $\O(|\calI|\cdot
S^m)$.

Nonetheless, we are convinced that our approach is much faster than
implementing \MaxFeas using integer linear programming or using a CSP
solver (which has been done in \cite{gabay2013lbv2}) and contributes
to the fact that we can solve much larger instances.
