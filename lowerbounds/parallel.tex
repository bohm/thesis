\section{Parallelization}\label{subsec:para}

Up until now, we have described a single-threaded minimax algorithm
with caching and pruning. We have used this single-threaded version in
\cite{bohm16} to improve on the results of Gabay, Brauner and Kotov
\cite{gabay2013lbv2} and reach a lower bound of $45/33 =
1.\overline{36}$ for $3$ bins and a lower bound of $19/14$ for up to
$5$ bins.

In the pursuit of an improved bound for $3$ bins as well as a general
lower bound for $m$ bins, we have subsequently implemented a parallel
version of the minimax search algorithm.

\paragraph{Technology.} We have settled on using a combination
of OpenMPI \cite{openmpi} and standard thread library as provided
by the C++ programming language. We have tried implementing


\paragraph{Tasks.} Our evaluation of the game tree proceeds in the
following way: First, we start evaluating the game tree on the main
computer (which we internally call \emph{queen}) until a vertex
corresponding to \adversary's next move meets a certain threshold (for
instance, sufficient depth). After that, we designate this adversarial
vertex as a \emph{task}.

Alongside the queen, we have processes whose job is to evaluate the
tasks -- we call them the \emph{workers}. Workers which run on the
same machine will have a common cache that they access via atomic
primitives in order to maintain consistency. Workers on separate
machines do not share information.

Due to the mixed environment of standard Unix threads and MPI
processes, we also have a single \emph{overseer} per each physical
machine. This overseer handles the MPI communication as well as
spawning the individual worker threads.

The tasks are all generated in advance by the queen. After that, their
bin configurations are synchronized with all overseers running. The
queen then assign tasks to overseers online, namely by assigning a
batch of 250-500 tasks to an overseer. The overseer reports each value
of a finished task immediately to the queen. When an overseer is
finished processing a batch, it requests and receives a new one.

We have selected this communication strategy for two reasons:

\begin{enumerate}
\item To minimize congestion in the processing phase through the fact
that the bin configurations are synchronized beforehand and only
identifiers are shared in the online assignment phase.
\item To allow the queen to evaluate and prune unfinished tasks and
therefore avoid some unnecessary processing by the workers.
\end{enumerate}

\paragraph{Task selection.}
\todo{Finish this paragraph.}

\paragraph{Saplings.} Our implementation also allows us to pre-select
some initial strategy for the player \adversary in advance. This way
we can use our (so far limited) intuitive understanding of what is a
good initial move and decrease the time needed to evalate the whole
tree.

A particularly good strategy for the lower bound of $19/14$ seems to
be sending an item of size $5$ as the first item, followed by $m-1$
items of size $1$. This adversarial strategy leads to a lower bound
instance for $6,7,8$ and $9$ bins.

\todo{Finish this paragraph.}


