\section{Caching}\label{sec:4:caching}

Our minimax algorithm employs extensive use of caching. We cache
solutions of the dynamic programming procedure \MaxFeas as well as any
evaluated bin configuration $C$ (as a hash) with its value.

\paragraph{Hash table properties.} We store a large hash table of
fixed size with each entry being a 64-bit integer corresponding to
$63$ bits of hash and a binary value. The hash table is addressed
by a prefix of the hash, usually between $20-30$ bits (depending
on the computer used).

We solve the collisions by a simple linear probing scheme of a fixed
length (say $4$). In it, when a value needs to be inserted to an
occupied position, we check the following $4$ slots for an empty space
and we insert the value there, should we find it. If all $4$ slots are
occupied, we replace one value at random.

\paragraph{Hash function.} Our hash function is based on Zobrist
hashing \cite{zobrist}, which we now describe.

For each bin configuration, we count occurrences of items, creating
pairs $(i,f)$ belonging to $\{1,\ldots,S\} \times \{0,1\ldots,m\cdot
S\}$, where $i$ is the item type and $f$ its frequency (the number of
items of this size packed in all $m$ bins).

As for the loads of the $m$ bins, we maintain that they are sorted in
descending order. We also think of them as ordered pairs $(j,g)$, with
$j$ being the position of the bin in the ordering (e.g. $1$ --
largest, $m$ -- smallest) and $g$ the actual value of the load.

For example, we can think of bin configuration
$((3,3,2),\{1,1,1,1,2,3\})$ as a set of load pairs $(1,3)$, $(2,3)$,
$(3,2)$ along with pairs for items: $(1,4)$, $(2,1)$, $(3,1)$,
$(4,0)$, $(5,0)$ and so on.

At the start of our program, we associate a $64$-bit number with each
pair $(i,f)$. We also associate a $64$-bit number for each possible
load of one bin. These two sets of numbers are stored as a matrix of
size $S \times (m\cdot S)$ and a matrix of size $(R-1) \times m$.

The Zobrist hash function is then simply a \texttt{XOR} of all
associated numbers for a particular bin configuration.

The main advantage of this approach is fast computation of new hash
values.  Suppose that we have a bin configuration $m$ with hash
$H$. After one round of the player \adversary and one round of the
player \algo, a new bin configuration $B'$ is formed, with one new
item placed.

Calculating the hash $H'$ of $B'$ can be done in time $\O(m)$,
provided we remember the hash $H$; the new hash is calculated by
applying XOR to $H$, the new associated values, and the previous
associated values which have changed.



\paragraph{Caching of the procedure \MaxFeas.} We use essentially the
same approach for caching results in the procedure \MaxFeas, except
only the $m$-tuple of loads needs to be hashed.

We also remark upon the values being cached in the procedure \MaxFeas.
At first glance, it seems that it might be best to store the value of
$y$ with each input multiset $\calI$. However, this is a very bad
idea, as we would lose upon a lot of symmetry.

Indeed, if we set $i$ to be any item from the list $\calI$, we would
lose out on the fact that we know a lower bound on the largest value
that can be sent for a multiset $\calI \setminus \{i\} \cup \{y\}$ --
namely $s(i)$, the value we know is compatible.

Instead, it is much better to cache binary feasibilities or
infeasibilities for a specific multiset $\calI$. We use these results
to improve the values of $LB$ and $UB$ for other calls of procedure
\MaxFeas.


