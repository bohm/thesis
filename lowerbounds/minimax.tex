\section{Minimax algorithm}\label{sec:4:minimax}
Let us recall the bin stretching game, as defined in the introduction:

\begin{dfn}
Fix a number of bins $m$ and a stretching factor $R$. The \emph{bin stretching game}
is defined as follows:

\begin{enumerate}
\item It is a two-player game between \algo and \adversary, and the player \adversary starts.
\item The player \adversary moves by sending any item $i \in (0,1]$ with the restriction
that $i$ along with all previous items can be packed into $m$ bins of capacity $1$.
\item The player \algo moves by selecting a bin where the item is packed.
\end{enumerate}

The winning states are defined thus:
\begin{itemize}
\item A state is winning for \algo if all bins are packed strictly below $R$ and \adversary
has cannot make any further moves.
\item A state is winning for \adversary if one bin has total load at least $R$.
\end{itemize}
\end{dfn}

Our main goal is learning whether there exists an algorithm for
\binstretch with stretching factor strictly below $R$, or whether
every algorithm needs a stretching factor of at least $R$. This
directly corresponds to existence of a winning strategy for the player
\algo or \adversary, respectively.

A ``gold standard'' algorithm for evaluating a given combinatorial
two-player game is the \emph{minimax algorithm}, which we can
summarize as follows:

\begin{algorithm}
\caption{Algorithm \textsc{Minimax}$(s)$ for a state $s$}
% \label{alg:bounded}
\begin{algorithmic}[1]
\State \algorithmicif\ the state is defined as winning for any player, return.
\For{every move $m$ of the current player}
\State Construct the state $t$ by applying the move $m$ to the current state $s$.
\State Run \textsc{Minimax}$(t)$.
\State \algorithmicif\ $t$ is winning for the current player, return.
\EndFor
\State \Return the fact that the state $s$ is winning for the other player.
\end{algorithmic}
\end{algorithm}


As we also mentioned in the introduction, the two main obstacles to
implementing \minimax for \binstretch are the following:

\begin{enumerate}
\item \adversary can send an item of arbitrary  size;
\item \adversary needs to make sure that at any time of the game, an offline
  optimum can pack the items arrived so far into $m$ bins of capacity $1$.
\end{enumerate}

To overcome the first problem, it makes sense to create a sequence of
games based on the granularity of the items that can be packed.  The
second problem increases the complexity of every game turn of the
\adversary, as it needs to run a subroutine to verify the guarantee
for the next item it wishes to place.

