\section{Description of the finite game}

We now formulate precisely the scaling parameter and other main
concepts of the game we wish to solve. First of all, for scaling
purposes, we rescale the guaranteed load of the optimum packing from
$1$ in the original definition to $S \in \Nat$. Therefore, if the
algorithm must fill at least one bin to capacity $R$, the lower bound
on the stretching factor will be $R/S$.

Next, we formalize what will be one state of our game:

\begin{dfn}\label{dfn:binconf}
  Assume we fix three integral parameters: $m$ (the number of bins), $R$ (capacity of bins of the algorithm)
and $S$ (size of the optimal bin). Then, we define a \emph{bin configuration} to be a pair
  $(\calL,\calI)$, where
  \begin{compactitem}
    \item $\calL$ is a tuple of $m$ non-negative integers, sorted in descending order. These denoting the current sorted loads of the bins.
    \item $\calI$ is a multiset with ground set $\{1,2,\ldots,S\}$ which contains the items used in the bins.
  \end{compactitem}

  Additionally, for every bin configuration, it must hold:
  \begin{compactitem}
    \item that there exists a packing of items from $\calI$ into $m$ bins with loads exactly in $\calL$;
    \item that there exists a packing of items from $\calI$ into $m$ bins that does not exceed capacity $S$ in any bin.
  \end{compactitem}  
\end{dfn}

To illustrate the definition, consider a configuration
$((4,3,2,0),\{1,1,2,2,3\})$ -- a situation where four items arrived,
one bin is loaded to size $4$, one to size $3$ and one to size $2$,
leaving one more bin empty. Notice that the bin configuration does not
capture the entire history of the game, but only its current state --
indeed, we do not know in which order the items arrived, or even
whether the bin of size $4$ is filled as an item of size $3$ plus an
item of size $1$, or $2$ plus $2$.

Losing information about history might seem like a bad idea at first,
but it becomes more advantageous later, as it makes caching more
efficient.

It is clear that every bin configuration is a valid state of the game
with \adversary as the next player. We may also observe that the
existence of an online algorithm for \binstretch implies an existence
of an oblivious algorithm with the same stretching factor that has
access only to the current bin configuration $m$ and the incoming item
$i$.

In the following easy observation, we make sure that bin configuration
captures all we need for both players:

\begin{obs} Fix parameters $m$, $R$ and $S$.
\begin{enumerate}
\item There exists a lower bound adversarial strategy for \binstretch if and only if there exists a strategy
that receives a bin configuration and returns the next item in the lower bound.
\item There exists an algorithm for \binstretch if and only if there exists an algorithm which receives
a bin configuration and a new item and returns a new bin configuration where the new item is packed.
\end{enumerate}
\end{obs}

We are now able to formally define the game we investigate:

\begin{dfn}
For a given $m \in \Nat, R \in \Nat$ and $S \in \Nat$, the \emph{bin stretching game} $\Game(m,R,S)$ is the following two-player game:

\begin{enumerate}
\item There are two players named \adversary and \algo. The player \adversary starts.
\item Each turn of the player \adversary is associated with a bin configuration $C = (\calL,\calI)$.
The start of the game is associated with the bin configuration $((0,0,\ldots,0),\emptyset)$.
\item The player \adversary receives a bin configuration $C$. Then,
\adversary selects an integer $i$ such that the multiset $\calI \cup
\{i\}$ can be packed by an offline optimum into $m$ bins of
capacity $S$. The pair $(C,i)$ is then sent to the player \algo.
\item The player \algo receives a pair $(C,i)$. The player \algo has
to pack the item $i$ into the $m$ bins as described in $C$ so that
each bin has load \emph{strictly less than} $R$. \algo then updates the
configuration $C$ into a new bin configuration, denoted $C'$. \algo
then sends $C'$ to the player \adversary.
\item If the player \algo receives a pair $(C,i)$ such that it cannot pack
the item according to the rules, the bin configuration $C$ is won for player \adversary.
\item If the player \adversary has no more items $i$ that it can send from a
configuration $C$, the bin configuration $C$ is won for player \algo.
\end{enumerate}
\end{dfn}

\begin{dfn}
We say that a game $\Game(m,R,S)$ is a \emph{lower bound} if the player
\adversary has a winning strategy when starting from the bin configuration
$((0,0,\ldots,0),\emptyset)$.
\end{dfn}
