\section{Monotonicity}\label{sec:monotonicity}

One of the new heuristics that enables us to go from a lower bound of
$19/14$ on $5$ bins to $9$ bins is iterating on lower bounds by
monotonicity. We define it as follows:

\begin{dfn}
A winning strategy for \adversary has \emph{monotonicity} $k$ if it is true
that for any two items $i,j$ such that $j$ is sent immediately after
$i$, we have $s(j) \ge s(i) - k$.
\end{dfn}

Using this concept, we can iterate over $k$ from $0$ (non-decreasing
instances) to $S-1$ (full generality) to find the smallest value of
monotonicity which leads to a lower bound, if any.

A potential downside of iterating over monotonicity is that it can
introduce an $S$-fold increase in elapsed time in the case that no
lower bound exists. Additionally, it is quite likely that monotonicity
becomes less useful as the scaling factor $T$ increases, as the item
of relative size $1$ gets smaller and smaller.

Still, solving decision trees of low monotonicity is much faster than
solving the full tree, and we have empirically observed that lower
bounds of lower monotonicity are fairly common; see
Tables~\ref{tab:results3} and \ref{tab:resultsmulti} for our empirical
results.

\paragraph{Monotonicity caveats.} There are two important notes
regarding monotonicity that need to be discussed. First, it is now
true that bin configuration is not enough to describe one state of the
bin stretching game. To see this, consider monotonicity $1$. If the
first three input items are $1,2,3$, the next item needs to be of size
$2$ or larger. However, if the three input items are $1,3,2$ (which is
permissible for monotonicity $1$), the next item on input can be of
size $1$ and above. This means that the two states are not equivalent,
even though their bin configuration is the same.

Notice that this problem is absent in the general case (where any item
can be sent at any time, assuming it satisfies the optimum packing
guarantee) and also in the case of monotonicity $0$ (where the last
item can be inferred from the bin configuration, as it is always the
item with the largest size).

To remedy this, we mark in the bin configuration which item arrived
last in the input sequence, which is sufficient for a fixed value of
the monotonicity.

When increasing monotonicity by one and re-running our algorithm, we
need to update our cached results (Section~\ref{sec:4:caching}) as
some previously losing bin configurations may become winning for the
player \adversary with a less monotone input. We could erase the
entire cache, but it is not necessary -- as any feasible bin
configuration stays feasible and any bin configuration that is winning
for \adversary stays winning when monotonicity increases. Therefore,
we only erase the infeasible configurations and the bin configurations
winning for the player \algo.

The second caveat concerns combining other adversarial pruning
(defined later in Section~\ref{sec:4:advpruning}) with monotonicity.

When pruning the game tree, we enable all adversarial pruning tests
including those that send items that would break the monotonicity
constraints.

There is no structural problem when doing so, since monotonicity is
only restricting \adversary to reduce the tree size; giving \adversary
some power back is therefore allowed. However, we must take this into
account when reading the results of Tables~\ref{tab:results3} and
\ref{tab:resultsmulti}. For example, a lower bound of monotonicity $1$
may actually allow sending any item in the last few steps of the
instance, provided this leads immediately to \algo packing one bin
fully to its capacity $S$.

