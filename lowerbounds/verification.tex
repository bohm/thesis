\newpage
\section{Verification}\label{sec:4:verification}

For our lower bounds of $19/14$ for $4 \le m \le 8$ we do not have a
printable version of the game tree, as it contains too many vertices.

However, we include the lower bounds in the \texttt{graphviz}
representation as part of our software appendix. We have published the
lower bound software and the lower bound data at
\url{https://github.com/bohm/binstretch/}.

Our lower bound implementation is quite large in terms of number of
lines of code (8174 lines) and we cannot in good faith vouch for it to
not contain any errors. At the same time, we do not expect the reader
to check all the lower bound trees by hand, even though it is possible
for a smaller tree such as the one in the next section.

To that end, we have implemented a simple independent C++ program
which verifies that a given game tree is valid winning strategy for
the player \adversary and that all possible options of player \algo
are present. While verifying our lower bound manually may be
laborious, verifying the correctness of the C++ program should be
manageable. The verifier is available along with the lower bound
search code and lower bound data.


\subsection{Game tree format}\label{sec:4:outputformat}

We now describe our output format for the lower bound directed acyclic
graph that our implementation produces. This may be of use to those
who wish to write an independent verification software.

The output format is a valid \texttt{graphviz} file format, which
means it can be converted directly into a graphical representation
using the open-source graphviz library using a command such as:
\begin{lstlisting}[language=bash,numbers=none]
$ dot -Tpdf 45_33_3bins.dot > 45_33_3bins.pdf
\end{lstlisting}
The starting lines of the lower bound tree are the following:

\begin{lstlisting}
strict digraph sapling1 {
overlap = none;
// 6: 5 1 1 1 1 1
\end{lstlisting}

The third line is technically a comment in the \texttt{graphviz}
format; it contains the number of items already packed into the root
of the game tree (or sapling, see Section~\ref{sec:4:advpruning}).  The
number of items is then followed by an enumeration of such items.

The rest of the graph is the description of vertices (game states) and
edges (transitions between game states). Every vertex in the output
graph corresponds to a game state for the player \adversary; the game
states for the player \algo correspond to outgoing edges.

A typical vertex representation (in this case, a root) might be:

\begin{lstlisting}
66 [label="5 2 1 1 1 0 0 0 n:4"];
66 -> 41530
66 -> 41529
66 -> 41528
66 -> 41248
\end{lstlisting}

The first number on each line is an \emph{id} of the vertex. As these
come from our lower bound algorithm (which will inevitably prune some
vertices), the numbers might not be contiguous or starting from
0.

Notice that the root can also have an arbitrary id number as well; we
assume that the root is always the first listed vertex in any output
tree.

The first line of the vertex defines the loads of the bins of player
\algo along with the next item to be sent (4 in our case).

The following lines define all outgoing edges from the vertex; as
mentioned above, these correspond to the possible packings of the next
item by the player \algo. It is guaranteed that the target vertex for
every edge exists in the output description.

If the vertex is a heuristic vertex, its description may be one of the
following:

\begin{lstlisting}
107982 [label="9 5 5 4 4 4 2 1 h:FN (1)"];
107981 [label="9 5 5 5 4 4 2 0 h:(14,5)"];
\end{lstlisting}

These are the only two adversarial pruning heuristics present in the
tree; see Section~\ref{sec:4:advpruning} for their specification. In
the example above, the next move for the first vertex describes a
\emph{five-nine} heuristic that will send 1 item of size 5 and after
that items of size either 9 or 14, as explained in
Section~\ref{sec:4:advpruning}).

The strategy listed in the vertex on line 2 is the \emph{large item
heuristic}.  The two numbers describe the size of the incoming items
and their amount. In our example, we see that sending 5 items of size
14 will force one bin to be loaded to at least 19.

As can be expected, the final line of the input is closing the curly
bracket at the start.

Our output format is inherently designed for converting to a graphical
form. A consequence of that is that there are no item lists in any of
the vertices, only the next item that will be sent. Any verification
algorithm is required to reconstruct the item lists from the tree
structure.