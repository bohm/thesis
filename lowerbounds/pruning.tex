\section{Tree pruning}\label{sec:4:pruning}

Alongside the extensive caching described in Subsection
\ref{sec:4:caching}, we also prune some bin configurations where it is
possible to prove that a simple online algorithm is able to finalize
the packing. Such a bin configuration is then clearly won for player
\algo, as it can follow the output of the online algorithm.

\subsection{Algorithmic pruning}\label{sec:4:gs}

Recall that in the game $\Game(m,R,S)$, the player \algo is trying to
pack all items into $m$ bins with load at most $R-1$. If the search
algorithm can quickly deduce that a bin configuration leads to a
successful packing, we can immediately evaluate the configuration as
winning for the player \algo and thus prune the tree.

This may remind us of the \emph{good situations} of
Chapter~\ref{chap:threebins} (specifically Section~\ref{sec:3:gs}).
Indeed, in the case of $3$ bins, we can use the good situations of
Section~\ref{sec:3:gs} directly.

For general $m$, we restate some of the good situations for a general
instance of $\Game(m,R,S)$ such that $(R-1)/S \ge 1/3$. Recall that $\alpha$
is the extra space in a stretched bin; in the situations below, $\alpha = (R-1) -S$,
as the player \algo is packing only to capacity $R-1$.

We omit the proofs of the good situations; they follow the same
arguments as in the proofs in Section~\ref{sec:3:gs}.

% \todo{I feel bad for omitting them, but time doesn't seem to be on my side.}

\setcounter{goodsit}{0}

\begin{goodsit}\label{lem:gs1generic}
Given a bin configuration $(\calL,\calI)$ such that the total load
of all but the last bin is at least $(m-1)\cdot S - \alpha$,
there exists an online algorithm that packs all remaining items into
$m$ bins of capacity $R-1$.
\end{goodsit}

\begin{goodsit}\label{lem:gs2generic}
Given a bin configuration $(\calL,\calI)$ such that there exist
two bins $A,B$ such that $s(A) \le \alpha$ and
$s(\calL \setminus \{A,B\}) \ge (m-2)\cdot S - 2\alpha - 1 $,
there exists an online algorithm that packs all remaining items into
$m$ bins of capacity $R-1$.
\end{goodsit}

% \todo{Fix good situations 3 and 4.}

\begin{goodsit}\label{lem:gs3generic}
Consider a bin configuration $(\calL,\calI)$. Define the following
sizes:

\begin{enumerate}
\item Let $s$ be the sum of loads of all bins excluding the last two.
\item Let $r$ (the \emph{last bin load requirement}) be
the smallest load such that if the currently last bin $B_m$ has load at least $r$,
GS\ref{lem:gs1generic} is reached (after reordering the bins).
\item Let $o$ (the \emph{overflow}) be defined as $R-r$.
\end{enumerate}

Then, if $r \le R-1$ and:

\begin{itemize}
\item either the second-to-last bin $B_{m-1}$ has load at most $\alpha$
and above $(m-1)\cdot S - \alpha - o - s$;
\item or the last bin $B_m$ has load at most $\alpha$
but above $(m-1)\cdot S - \alpha - o - s$;
\end{itemize}

there exists an online algorithm that packs all remaining items into
$m$ bins of capacity $R-1$.
\end{goodsit}

\begin{goodsit}\label{lem:gs4generic}

Suppose that for a bin configuration $(\calL, \calI)$ it holds that
the last two bins are loaded up to $\alpha$. 

Define the following sizes:

\begin{enumerate}
\item Let $r$ be the last bin load requirement as above.
\item Let $c$ (\emph{critical value for $B_{m-1}$}) be equal to $r - s(B_{m-1})$.
In other words, if an item of size in $[c,S]$ is packed into $B_{m-1}$, GS\ref{lem:gs1generic}
is reached.
\end{enumerate}

Then, for every item size in the interval $[1,c-1]$, we check the following
constraint:

Then, if it is true that for every item size in the interval $[1,c-1]$
we can pack enough of these items into $B_{m}$ such that
$GS\ref{lem:gs1generic}$ is reached on all bins except $B_{m-1}$,
there exists an online algorithm that packs all remaining items into
$m$ bins of capacity $R-1$.

\end{goodsit}

Interestingly, the last good situation (Good Situation
\ref{lem:gs4generic} above) has no fixed threshold like its original
form (GS\ref{lem:gs4}) but is \emph{algorithmic} in the sense that we
can check it for each item size separately with an algorithm and gain
a good situation if all its conditions are met.

\subsection{Adversarial pruning}\label{sec:4:advpruning}

Compared to our fairly strong algorithmic pruning, we have only few
tools to quickly detect that a bin configuration is winning for the
player \adversary. More specificaly, we use only the following two
criteria:

\paragraph{Large item heuristic.} Once any bin has load at least $R-S$,
an item of size $S$ packed into that bin would cause it to reach load
$R$, which is a victory for the player \adversary. Suppose that the
$k$-th bin reaches load $l \ge R-S$. We compute the size of the
smallest item $i$ such that

\begin{enumerate}
\item $s(i) + l \ge R$;
\item For any bin $b$ in the interval $[(k+1), m]$ it holds
that $s(i) + 2l \ge R$; in other words, \algo cannot pack
two items of size $l$ into any bin starting from the $(k+1)$-st.
\end{enumerate}

Finally, we check if \adversary can send $m-k+1$ copies of the item of
size $l$. If so, it is a winning bin configuration for this player and
we prune the tree.

Notice that there may be multiple different values of $l$ for one bin
configuration; for instance, in the setting of $19/14$, for three bins
with loads $11,9,0$, we should check whether we can send $2$ items of
size $10$ or $3$ items of size $8$. Therefore, in the implementation,
we compute for each bin its own candidate value of $l$ and then check
whether at least one is feasible using the dynamic programming test
described in Section~\ref{subsec:test}.

\paragraph{Five/nine heuristic.} We use a specific heuristic for the
case of $19/14$, as it is a good candidate for a general lower
bound. This heuristic was observed to slightly compress the size of
the output tree in this setting.

This heuristic comes into play once there is a bin of load at least
$5$. The item sizes $5$ and $9$ are complementary in the sense that
one of each can fit together in the optimal packing of capacity 14,
but the two of them cannot be packed together into a bin that already
has load at least $5$. Additionally, if there are too many bins of
load at least $5$ but not much more, a subsequent input of several
items of size $14$ will again force a bin of load at least $19$.

We apply this heuristic only when it is true that at all times, $m$
items of size $9$ can arrive on input without breaking the adversarial
guarantee. While this is true, it must be true that all bins are of
load strictly less than $10$.

Our heuristic considers repeatedly sending items of size $5$. If at
any point there are only $p$ bins left with load strictly less than
$5$ and at the same time $p+1$ items of size $14$ can arrive on input,
the configuration is winning for the player \adversary. On the other
hand, if at any point there is a bin of load at least $10$ and the
invariant that $m$ items of size $9$ can still arrive holds, we are
also in a winning state for \adversary.

If it is true that by repeatedly sending items of size $5$ we
eventually reach at least one of the aforementioned two situations,
we mark the initial bin configuration as winning for the player
\adversary.

\paragraph{A note on performance.} While both of our heuristics reduce
the number of tasks in our tree and the number of considered vertices,
we were unable to evaluate them in every single vertex of the game
tree without a performance penalty. Even the large item heuristic,
which can be implemented with just one additional call to the dynamic
programming procedures of Section~\ref{subsec:test} slows the program
down considerably.

This is likely due to the fact that caching outputs of the dynamic
programming calls of Section~\ref{subsec:test} leads to some vertices
that do not need to call any dynamic programming procedure, and with
our heuristics they are forced to call at least one.

% We explain the details of our caching strategies in the following
% section.