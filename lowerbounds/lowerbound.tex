\openright
\chapter{Computer lower bounds}\label{chap:lb}

% potential todo: why are lower bounds so hard?

In this chapter, we focus on the algorithmic approach to computing
lower bounds for \binstretch. We explain the technique of Gabay,
Brauner and Kotov~\cite{gabay2013lbv2} as well as our extensions and
its implementation.

\paragraph{An intermezzo with an anecdote.} Before we continue with
the explanation of the central ideas of our computer search, let us
pause to explain the motivation of the thesis author to work on the
lower bound algorithms.

While working together with Jiří Sgall, Rob van Stee and Pavel Veselý
on the algorithmic results of Chapters~\ref{chap:manybins} and
\ref{chap:threebins}, we have also investigated lower bounds for
\binstretch. With some effort, the author of this thesis was able to
prove a lower bound of $4/3 + \varepsilon$ for $3$ bins and a very
small value of $\varepsilon$.

To our surprise, another paper on \binstretch appeared while we were
working on the problem, namely the one by Gabay, Brauner and Kotov
\cite{gabay2013lbv2} showing a much better lower bound of $19/14$
using a clever minimax computer search and using a CSP solver to
verify the optimum guarantee in every step. Gabay et al. were able to
check all integer denominators up to $20$, after which their computer
program ran out of memory.

In this paper of Gabay et al. (but not in its current,
updated version) we could find the following fateful sentence:

\textit{``The combinatorial explosion is very well illustrated in column
\texttt{\#nodes} where we can see that even with many efficient cuts,
we cannot tackle much larger problems.''} \cite{gabay2013lbv2}

Even though the sentence itself might be considered incorrect (seeing
as we \emph{can} tackle much larger problems, and in fact we will
tackle problems up to denominator $82$), the author of this thesis
considers it quite ingenious. That one sentence turned a reasonably
good result into a challenge and ignited the author's spark for
computer lower bounds.
