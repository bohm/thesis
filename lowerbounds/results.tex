\section{Results}\label{sec:results}

Tables~\ref{tab:results3} and \ref{tab:resultsmulti} summarize our
results. The paper of Gabay, Brauner and Kotov \cite{gabay2013lbv2}
contains results up to the denominator 20; we include them in the
table for completeness. Results after the denominator 20 are new. Note
that there may be a lower bound of size say $\frac{41}{30}$ even
though none was found with this denominator; for example, some lower
bound may reach $41/30$ using item sizes that are not multiples of
$1/30$.

\paragraph{A note on 9 bins.} We have produced a verifiable lower
bound of $19/14$ for the setting of 4-8 bins. The setting of $9$
bins lies at the threshold of the computational power of our
implementation -- we have been able to successfully run the lower
bound search without producing any output for $m=9$, but as of the
submission date we were not able to generate the verifiable lower
bound tree.

Still, based on estimating the partial progress of the algorithm, we
are convinced that we will be able to produce a verifiable lower bound
of $19/14$ for $9$ bins within a few days. If that happens, we will
update our data set at \url{https://github.com/bohm/binstretch/} to
include it as well.

\begin{table}[H]
\begin{center}
% \begin{tabular}{ c | @{\hskip 2em} l | @{\hskip 2em} l | @{\hskip 2em} l }\label{tab:results}
\begin{tabular}{ llllll }
 & & & &  \multicolumn{2}{c}{\textit{Elapsed time}}  \\
\textit{Fraction} & \textit{Decimal} & \textit{L. b.} & \textit{Mon.} & \textit{Linear} & \textit{Parallel}\\

\hline
$19/14$ &  $1.3571$ & Yes & 0 & 2s. & \\
$22/16$ & $1.375$ & No & & 2s. & \\
$26/19$ & $1.3684$ & No & & 3s. & \\
\hline
$30/22$ & $1.\overline{36}$ & No & & 6s. & \\
$33/24$ & $1.375$ & No & & 5s. & \\
$34/25$ & $1.36$ & \textbf{Yes} & 1 & 15s. & \\
$45/33$ & $1.\overline{36}$ & \textbf{Yes} & 1 & 1min. 48s. & \\
$55/40$ & $1.375$ & No & & 3min. 6s. & \\
$56/41$ & $1.3659$ & No & & 30min. & 7s. \\
$86/63$ & $1.36507$ & \textbf{Yes} & 6 & & 29s. \\
$112/82$ & $1.3659$ & \textbf{Yes} & 8 & & 3h. 21m. 31s.\\
\end{tabular}
\end{center}
\caption[LoF entry]{The results and performance of our linear and
parallel computations for \binstretch with three bins. The results
above the horizontal line were previously shown in \cite{gabay2013lbv2}, the
rest are our results. The column \textit{L. b.} indicates whether a
lower bound was found when starting with the given stretching factor
$R/S$ as seen in column \textit{Fraction}.

The column \textit{Mon.} shows the lowest monotonicity that our
program needs to find a lower bound. In the case of negative results,
time measurements were done only using full generality, i.e. with
monotonicity $S-1$.

Some fractions below $112/82$ are omitted; our lower bound computation
has not found a lower bound on those.

The linear results were computed on a server with an AMD Opteron 6134
CPU and 64496 MB RAM. The size of the hash table was set to $2^{25}$.

The parallel results were computed using OpenMPI on a heterogenous
cluster with $109$ worker processes running.

The output of the program was not generated during the
time measurements.}
\label{tab:results3}
\end{table}

\begin{table}[H]
\begin{center}
\begin{tabular}{lllllll}
& & & & & \multicolumn{2}{c}{\textit{Elapsed time}}  \\
\textit{Bins} & \textit{Fraction} & \textit{Decimal} & \textit{L. b.} & \textit{Mon. (5)} & \textit{Linear} & \textit{Parallel (5)}\\
\hline
$4$  & $19/14$ &  $1.3571$ & \textbf{Yes} & & & 18s.  \\
$4$  & $30/24$ & $1.\overline{36}$ & No   & & & 19s. \\
$4$  & $34/25$ &  $1.36$   & No           & & & 48s.  \\ 
$5$  & $19/14$ &  $1.3571$ & \textbf{Yes} & 2 (1) & & 10s. \\
$6$  & $19/14$ &  $1.3571$ & \textbf{Yes} & 0 (0) & & 11s. \\
$7$  & $19/14$ &  $1.3571$ & \textbf{Yes} & 1 (0) & & 2m. 13s. (16s.) \\
$8$  & $19/14$ &  $1.3571$ & \textbf{Yes} & Unk. (1) & & (1h. 14s.)  \\
% $9$  & $19/14$ &  $1.3571$ & \textbf{Yes} & Unk. (1) & &  \\
\end{tabular}
\end{center}
\caption{Results produced by our minimax algorithm for more than $3$ bins.
Tested on the same machine and with the same parameters
as in Table~\ref{tab:results3}, both for linear and parallel
computations. In columns \textit{Mon.} and \textit{Parallel}, we list
in brackets monotonicity and elapsed time of computation for an input
having an item of size 5 at the start. Monotonicity is measured only
starting with the second item.}
\label{tab:resultsmulti}
\end{table}


