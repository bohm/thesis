\chapter{Conclusion}\label{chap:conclusion}
% \addcontentsline{toc}{chapter}{Conclusion}

\section{Summary of results}

We have focused single-mindedly on the online problem \binstretch
which concerns itself with packing items into $m$ bins of size $R$
with additional knowledge that there exists some packing of the entire
input into $m$ bins of capacity $1$. The problem is interesting by
lacking any non-trivial general lower bounds, even though reasonably
simple algorithms for it exist.

On the algorithmic side, we have seen a combination of the
\emph{classification and bunching} approach to design an online
algorithm for any number of bins $m$ with stretching factor $1.5$
(Chapter~\ref{chap:manybins}).  The analysis of our algorithm is
tight, but we suspect that even lower stretching factor can be reached
probably using different tools than just classification and bunching.

We have also shown a specialized algorithm for the case of the setting
where there are always only three bins. This algorithm achieves a
stretching factor $11/8 = 1.375$ by carefully balancing bins and
reaching good situations as soon as possible.

Finally, we have designed and implemented a computer program which is
able to search for lower bounds for \binstretch. We have built on the
core idea of~\cite{gabay2013lbv2} but we add a plethora of new tools
and implementation tricks which allowed us to far surpass the results
of the previous teams, including a parallel implementation that can be
run on a cluster of physical computers.

Using results from all three sections we have tightened the bounds on
the best possible value of the stretching factor:

\begin{itemize}
\item For the case of arbitrary many bins, the lower bound/upper bound interval is now $[1.\overline{3}, 1.5]$; 
\item For the case of $m = 3$, it is $[1.369, 1.375]$;
\item For the case of $4 \le m \le 8$, we have a new lower bound of $1.357$.
\end{itemize}

Unfortunately (or fortunately for the future researchers in the area),
the optimal stretching factor value for any fixed number of bins $m
\ge 3$ still remains unknown.

\section{What next?}

We now move on to discuss several open questions related to
\binstretch and problem areas which the author of this thesis believes
to be interesting. If you are interested in any one of them, feel free
to contact the author of this thesis for collaboration or for
answering any questions.

\subsection{The low-hanging fruit}

First, let us list a few problems which we believe can be achieved in
a month or two of work by an undergraduate or a graduate student.

\begin{enumerate}

\item \textit{Use more computing power to get better lower bounds.}
Our choice of a cluster of 109 cores was limited by the processing
capabilities of our institute. Additionally, several of the used
machines were desktops and we did not want to use them longer than
overnight when they were not in use by colleagues.

\noindent\textbf{Why isn't it included in this thesis?} We needed to stop
tinkering with the lower bound search program and start writing the
thesis at some point. We are quite convinced that one can get tighter
bounds on the stretching factor for $3 \le m \le 10$ by adding more
processing power or a few more pruning rules.

\item \textit{Determine the exact stretching factor for three bins.}
We now know that the best stretching factor achieved by any algorithm
for \binstretch lies somewhere in the interval $[1.3659,1.375]$. It
would be nice to discover the best algorithm for this setting as well
as a matching lower bound.

It is always hard to guess what the right number should be, but based
on the sequence of positive lower bound results, a good candidate (in
the author's opinion) for the optimal stretching factor is
$\frac{41}{30} = 1.3\overline{6}$.

\noindent\textbf{Why isn't it included in this thesis?} To design an
algorithm with a stretching factor below $1.375$ will probably require
either a new good situation or a fresh look at the algorithm design
for \binstretch on $3$ bins, perhaps using something other than the
good situations that we introduce.

We were originally hopeful that a lower bound of $\frac{41}{30}$ can
be found within a reasonable time, but since we were not able to show
it, we have not tried improving the algorithmic part either.
\end{enumerate}

\subsection{The sweeter fruit}

We now list open questions which, in our opinion, are much more
interesting to the online algorithm community as a whole, even though
their solution probably requires more time or manpower compared to the
ones above.

\begin{enumerate}

\item \textit{Design a lower bound higher than $4/3$ for any number of
bins.} It is still true that the lower bound of $4/3$ is the best
known lower bound for the stretching factor of \binstretch with $m$
bins ($m$ being part of the input). Our experimental results seem to
indicate that $19/14$ is a good candidate, as it holds for $m \in
\{3,4,\ldots,8\}$. 

\noindent\textbf{Why isn't it included in this thesis?} This problem was the
main motivation for our extensions to the lower bound search
algorithm. As it stands, the fact that we can get results for $m \le
8$ is promising, but we need many more adversarial cuts and
simplifications of the game tree before the lower bound tree for $m =
8$ became human-readable. Even then, we expect more fresh ideas to be
needed before the general lower bound can be shown.

\item \textit{Apply the same lower bound minimax approach to more
problems.} We are not convinced that \binstretch is the only online
problem where computer search can help finding better lower bounds.
It can be argued that many other online problems are notorious for the
configuration space being very large and no reasonable simplification
being known.

\noindent\textbf{Why isn't it included in this thesis?} Simply put, we have
not yet found the right problem where our lower bound search technique
would clearly produce new lower bounds. We will keep this technique in
mind for other problems, and we kindly ask the reader to keep an eye
for an application of this lower bound technique as well.

\item \textit{Can machine learning help design good online
algorithms?} There is much excitement in the broader programming
community regarding the recent progress of machine learning / deep
learning in designing winning strategies for positional and other
two-player games (e.g. chess or poker).

Already at the very beginning of this thesis (in
Section~\ref{sec:1:game}) we have observed that every online algorithm
corresponds to a strategy for a two-player game between \algo and
\adversary. Our questions are: Can we use the recent improvements to
machine learning to design either algorithms that perform well for an
online problem such as \binstretch? Also, can we use the good (if not
perfect) strategies produced by the machine learning algorithms to
prune a lower bound tree faster, similar to what we do in
Section~\ref{sec:4:pruning}?

% \todo{Citations.}
\noindent\textbf{Why isn't it included in this thesis?} We did not
have enough machine learning/AI expertise to confidently answer this
question in a positive or a negative way. We still believe it may be
possible, but it certainly requires a researcher well-versed in
current machine learning techniques.

\end{enumerate}