\chapter{Algorithmic Results for Many Bins}\label{chap:manybins}

% \addcontentsline{toc}{chapter}{Algorithmic Results for Many Bins}

This chapter describes an algorithm for \binstretch that achieves a
stretching factor of $1.5$ and works for any number of bins $m$. The
algorithm follows the \emph{classify and batch} approach as explained
in Section~\ref{sec:1:history}; this approach has been already used in
previous works.

Our main novel technique is applying the amortized analysis approach
to the batching arguments. With that and other improvements, our
algorithm is able to reach the stretching factor of $1.5$. The
goal of this chapter is to prove the following:

\begin{thm}\label{thm:onepointfive}
There exists an algorithm for \binstretch with a stretching factor of
$1.5$ for an arbitrary number of bins.
\end{thm}

As explained in Section~\ref{sec:1:binstretchdef}, we rescale item
sizes and bin capacities to avoid some fractional values in our
arguments. Specifically, we scale the input so that the optimal bins
have capacity 12 and the bins of the algorithm have capacity 18.


\section{Algorithm overview}\label{chap:2:overview}

In the first phase of the algorithm we try to fill the bins so that
their size is at most 6, as this leaves space for an arbitrary item in
each bin. Of course, if items larger than 6 arrive, we need to pack
them differently, preferably in bins of size at least 12 (since that
is the size of the optimal bins). We stop the first phase when the
number of non-empty bins of size at most 6 is three times the number
of empty bins. In the second phase, we work in blocks consisting of
three non-empty bins and one empty bin. The goal is to show that we
are able to fill the bins so that the average size is at least 12,
which guarantees we are able to pack the total size of $12m$ which is
the upper bound on the size of all items.

The limitation of the previous results using this scheme was that the
volume achieved in a typical block of four bins is slightly less than
four times the size of the optimal bin, which then leads to bounds
strictly above $3/2$. This is also the case in our algorithm: A
typical block may have three bins with items of size just above 4 from
the first phase plus one item of size 7 from the second phase, while
the last bin contains two items, each of size 7, from the second
phase---a total of 47 instead of desired $4 \cdot 12$. However, we
notice that such a block contains five items of size 7 which the
optimum cannot fit into four bins. To take an advantage of this, we
cannot analyze each block separately as
in~\cite{kellerer2013,gabay2013}.  Instead, the rough idea of our
improved method is to show that a bin with no item of size more than 6
typically has size at least 13 and amortize among the blocks of
different types. Technically this is done using a weight function $w$
that takes into account both the total size of items and the number of
items larger than 6. This is the main new technical idea of our proof.

There are other complications. We would like to ensure that a typical bin
of size at most 6 has size at least 4 after the first phase. However,
this is impossible to 
guarantee if the items packed there have size between 3 and 4. Larger
items are fine, as one per bin is sufficient, and the smaller ones are
fine as well as we can always fit at least two of them.
It is crucial to consider the items with sizes between 3 and 4 very 
carefully.
This motivates
our classification of items: Only the \emph{regular items} of size in
$(0,3]\cup(4,6]$ are packed in the bins filled up to size 6. The
\emph{medium items} of size in $(3,4]$ are packed in their own bins (four or
five per bin). Similarly, \emph{large items} of size in $(6,9]$ are
packed in pairs in their own bins. Finally, the \emph{huge items} of size
larger than $9$ are handled similarly as in the previous papers:
If possible, they are packed with the regular items, otherwise each in
their own bin.

The introduction of medium size items implies that we need to
revisit the analysis of the first phase and also of the case when the
first phase ends with no empty bin. These parts of the proof are
similar to the previous works, but due to the new item type we need to
carefully revisit it; it is now convenient to introduce another 
function $v$ that counts the items according to their type; we call it
a value, to avoid confusion with the main weight function $w$. The
analysis of the second phase when empty bins are present is more
complicated, as we need to take care of various degenerate cases, and
it is also here where the novel amortization is used.

% Now we are ready to proceed with the formal definitions, statement of
% the algorithm, and the proof our main result.
