
\section{First-phase algorithm}

During the algorithm, let $e$ be the current number of empty bins and
$r$ the current number of regular bins. 

\algobox{{\bf First-phase algorithm:}
\begin{compactenum}[(1)]
\item While $r < 3e$, consider the next item $i$ and pack it as
  follows, using bins of capacity 18; 
if more bins satisfy a condition, choose among them arbitrarily:
\item \indentskip If $i$ is regular:
\item \indentskip \indentskip If there is a huge-item bin, pack $i$ there.
\item \indentskip \indentskip Else, if there is a regular bin $A$ 
with $s(A)+s(i)\leq 6$, pack it there.
\item \indentskip \indentskip Else, if there is a tiny bin $A$
with $s(A)+s(i)\leq 6$, pack it there.
\item \indentskip If $i$ is medium and there is a medium-item bin where $i$ fits, pack it there.
\item \indentskip If $i$ is large and there is a large-item bin where $i$ fits, pack it there.
\item \indentskip If $i$ is huge:
\item \indentskip \indentskip If there is a regular bin, pack $i$ there.
\item \indentskip \indentskiptwodigit Else, if there is a tiny bin, pack $i$ there.
\item \indentskiptwodigit If $i$ is still not packed, pack it in an
  empty bin.
\end{compactenum}
}

First we observe that the algorithm above is properly defined.  The
stopping condition guarantees that the algorithm stops when no empty
bin is available. Thus an empty bin is always available and each item
$i$ is packed.  We now state the basic properties of the algorithm.

\begin{lem}
\label{l:1} 
At any time during the first phase the following holds:
\begin{compactenum}[\rm(i)]
\item 
\label{i1:classify} 
All bins used by the algorithm are of the types from Definition~\ref{d:types}.
\item 
\label{i1:complete} 
All complete bins $B$ have $v(B)\geq0$.
\item 
\label{i1:hr} 
If there is a huge-item bin, then there is no regular and no tiny bin.
\item 
\label{i1:lm} 
There is at most one large-item bin and at most one medium-item bin.
\item 
\label{i1:tiny} 
There is at most one tiny bin $T$. If $T$ exists, then for any regular
bin, $s(T)+s(R)>6$. There is at most one regular bin $R$ with
$s(R)\leq 4$.
\item 
\label{i1:er} 
At the end of the first phase $3e\leq r \leq 3e+3$.
\end{compactenum}
\end{lem}
\begin{proof}
(\ref{i1:classify})-(\ref{i1:tiny}): We verify that these invariants
  are preserved when an item of each type arrives and also that the
  resulting bin is of the required type; the second part is always
  trivial when packing in an empty bin.

If a huge item arrives and a regular bin exists, it always fits there,
thus no huge-item bin is created and (\ref{i1:hr}) cannot become
violated. Furthermore, the resulting size is more than $12$, thus the
resulting bin is complete. Otherwise, if a tiny bin exists, the huge
item fits there and the resulting bin is either complete or huge. In
either case, if the bin is complete, its value is 0 as it contains a
huge item.

If a large item arrives, it always fits in a large-item bin if it
exists and makes it complete; its value is at least 1, as it contains
two large items. Thus a second large-item bin is never
created and (\ref{i1:lm}) is not violated.

If a medium item arrives, it always fits in a medium-item bin if it
exists; the bin is then complete if it has size at least 13 and then
its value is at least 1, as it contains 4 or 5 medium items; otherwise
the bin type is unchanged. Again, a second medium-item bin is never
created and (\ref{i1:lm}) is not violated.

If a regular item arrives and a huge-item bin exists, it always fits
there, thus no regular bin is created and (\ref{i1:hr}) cannot become
violated. Furthermore, if the resulting size is at least $12$, the bin
becomes complete and its value is 0 as it contains a huge item; otherwise
the bin type is unchanged.

In the last case, a regular item arrives and no huge-item bin exists.
The algorithm guarantees that the resulting bin has size at most 6,
thus it is regular or tiny. We now proceed to verify
(\ref{i1:tiny}). A new tiny bin $T$ can be created only by packing an
item of size at most 3 in an empty bin. First, this implies that no
other tiny bin exists, as the item would be put there, thus there is
always at most one tiny bin. Second, as the item is not put in any
existing regular bin $R$, we have $s(R)+s(T)>6$ and this also holds
later when more items are packed into any of these bins. A new regular
bin $R$ with $s(R)\leq 4$ can be created only from a tiny bin; note
that a bin created from an empty bin by a regular item is either tiny
or has size in $(4,6]$. If another regular bin with size at most 4
  already exists, then both the size of the tiny bin and the size of
  the new item are larger than 2 and thus the new regular bin has size
  more than 4. This completes the proof of (\ref{i1:tiny}). 

(\ref{i1:er}): Before an item is packed, the value $3e-r$ is at least
  $1$ by the stopping condition. Packing an item may change $e$ or $r$ (or both) by at most $1$. Thus after
  packing an item we have $3e-r\geq 1-3-1=-3$, i.e., $r\leq 3e+3$. If
  in addition $3e\leq r$, the algorithm stops in the next step and
  (\ref{i1:er}) holds.
\end{proof}

\paragraph{Terminating the first phase. } If the algorithm packs all
input items in the first phase, it stops. Otherwise according to
Lemma~\ref{l:1}(\ref{i1:hr}) we split the algorithm in two very
different branches. If there is at least one huge-item bin, follow the
\emph{second phase with huge-item bins} below. If there is no huge-item bin,
we follow the \emph{second phase with regular bins}.

Any bin that is complete is not used in the second phase. In addition
to complete bins and either huge-item bins, or regular and empty bins,
there may exist at most three \emph{special bins} denoted and ordered
as follows: the large-item bin $L$, the medium-item bin $M$, and the
tiny bin $T$.

