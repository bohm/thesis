\subsection{The standard case}
From now on, we can assume that $s(j) < 4$, $j$ is packed into $B$ and
Step \step{qstep} of \tbalg is reached. Recall that by Observation~\ref{obs:1a}
$s(A_{\before j}) \in  (3,4)$, $s(B_{\before j}) + s(j) \in(6,8)$, and
there is exactly one item either in $A$, or in $B$; we denote this item by $e$.
We repeat the steps done by \tbalg in the standard case:

\smallskip
\algobox{
(\step{qstep}) \GSFF{$A|_4,B|_q,C|_4$} where
$q \defeq 9+\frac{1}{2}(s(A)+s(C))$. %\item[] \indentskip
Whenever $s(A)$ or $s(C)$ change in this step, update $q$ and continue Step~\step{qstep}.\\
(\step{b1}) \GSFF{$A|_4, B|_{22}, C|_{22}$}.\\
(\step{b2}) \GSFF{$A|_{22}, B|_{22}, C|_{22}$}.
}

Recall that $s(A) > 3$ by Observation~\ref{obs:1a}. Assuming
that no good situation is reached before Step \step{qstep}, we
observe the following:

% \smallskip

\begin{obs}
\label{obs:6-b}
In Step {\rm \step{qstep}}, as long as $C$ is empty,
packing any item of size at least $4$ leads to a good
situation. Thus while $C$ is empty, all items that 
arrive in Step {\rm \step{qstep}} and are not put on $A$ 
have size in $(6-s(A),4)$.
\end{obs}
\begin{proof}
Any item with size in $[4-s(A),6-s(A)]\cup[4,6]$ leads to \gs2. 
Any item with size more than 6 is assigned to $B$ if it fits there,
reaching \gs5, and else to $A$ or $C$, reaching \gs7 since $s(B)\le9+\frac{1}{2}(s(A)+s(C))$. The only remaining
possible sizes of items that are not packed into $A$ are $(6-s(A),4)$.
\end{proof}

\begin{cor}
\label{obs:c}
After Step {\rm \step{qstep}}, $C$ contains exactly one item $r$ and $s(A)+s(C)>6$.
\end{cor}

\begin{proof}
From the previous observation it is clear that $C$ receives at least one item $r$ in Step \step{qstep}. No second
item $r_2$ can be packed into $C|_4$ in Step~\step{qstep} as $s(r) + s(r_2)$ would be at least $2(6 - s(A)) > 4$.
\end{proof}


Step \step{qstep} terminates with a new item $x$ which fits into %PV: observation for further reference?
$B|_{22}$ (otherwise we would reach \gs7),
but not below the limit $q = 9 + \frac{1}{2}(s(A)+s(C))$. We pack $x$
into $B$ in Step~\step{b1}, getting $s(B) > 9 + \frac{1}{2}(s(A)+s(C)) > 12$.

A possible bad situation for our current packing is when three items $b_1,
b_2, b_3$ arrive, where the items are such that no two items
of this type fit together into any bin, and no single item of this
type fits on the largest bin, which is $B$ in our case. In fact, we
will prove later that this is the only possible bad situation.

We claim that this potential bad situation cannot
occur:

\begin{clm}\label{clm:nobigitems}
Suppose that the algorithm \tbalg reaches no good situation in the standard case. Then,
$s(C)\ge s(r)>2.8$ and after placing $x$ into B in Step~{\rm \step{b1}} it holds that
\[s(B)<12.8.\]

Furthermore, suppose that among items that arrive after $x$, there are three items $b_1, b_2, b_3$ such that 
\[\min( s(b_1), s(b_2), s(b_3)) > 8.\] Then, it holds that 
\[\min(s(b_1),s(b_2),s(b_3)) < 9.2. \]

% Martin: s(r) > 2.8 is not actually the best bound we can derive; later we can show that s(r)>3.
% However, s(r) > 2.8 is trivially true and it is enough for our calculations.
\end{clm}

We now show how Claim \ref{clm:nobigitems} finishes the analysis of \tbalg.
After that we show the claim using linear programming; a formal proof is in
Section~\ref{app:proofOfClaim}.

After Step \step{qstep}, assuming no good situation is reached, the
algorithm places $x$ into $B|_{22}$ and moves to Step~\step{b1},
which is \GSFF{$A|_4, B|_{22}, C|_{22}$}. Claim \ref{clm:nobigitems}
gives us $s(B) < 12.8$ after placing $x$, while the fact that we exited
Step \step{qstep} means that $s(B) > q = 9 + \left( s(A)+s(C) \right)/2 > 12$.

Consider the first item $b_1$ that does not fit into $A|_4$. We have
that $s(b_1)>2$, otherwise \gs2 is
reached. However, any item that fits into $B$ (as long as $s(C) \le 4$) triggers
\gs4, because $s(A) + s(B) + s(b_1) \ge  6 + 12 > 15 + s(C)/2$.

We now know that the first item $b_1$ does not fit into both $A|_{4}$ and $B|_{22}$.
We place it into $C$, noting that $s(b_1) > 22 - s(B) \ge  22 - 12.8 = 9.2$.

We keep packing items into $A|_{4}$, waiting for the second item $b_2$
that does not fit into $A|_4$ in Step~\step{b1}. Again, $s(b_2) >
2$. Suppose that $b_2$ fits into $B|_{22}$ or $C|_{22}$. Claim \ref{clm:nobigitems}
gives us $s(r)>2.8$; we thus sum up bins $B$ and $C$ and get $s(B_{\before b_2}) +
s(b_2) + s(r) + s(b_1) > 12 + 2 + 2.8 + 9.2 = 26$, which is enough for
\gs1.  Our assumption was false, the item $b_2$ does 
fit into neither $B|_{22}$ nor $C|_{22}$, in particular we have that $s(b_2)>9.2$.

We move to Step~\step{b2}, pack $b_2$ into $A|_{22}$ and continue
with \GSFF{$A|_{22}, B|_{22}, C|_{22}$}. If at any time $s(A) \ge  14$, we
enter \gs1 on $A$ and $B$. Otherwise, if an item $b_3$ does not fit into $A|_{22}$,
it must satisfy $s(b_3) > 8$.

We now apply the full strength of Claim \ref{clm:nobigitems}. The
smallest item of $b_1$, $b_2$, $b_3$ must have size less than $9.2$,
and because of our argument, it must be $b_3$ -- but this means it
fits into $B$, as $s(B)<12.8$. \gs1 on bins $A$ and $B$ finishes the
packing, since $s(A) + s(b_3) + s(B_{\before b_3}) > 22 + 12 > 26$.

\subsubsection{Proof of Claim \ref{clm:nobigitems}}\label{sec:profOfClaim}

Our current goal is to prove Claim \ref{clm:nobigitems}.
As in the large case, we would now like to appeal to the offline
layout of the larger items currently packed. Unlike the large case,
none of the items we have packed before Step~\step{b1} is guaranteed to be over $6$.

Sidestepping this obstacle, we will argue about the offline layout of
the smaller items. We now list several items that are packed before Step~\step{b2} and will be important in our analysis:

\begin{dfn}\label{dfn:fouritems} The four items $e,j,r,x$ are defined as follows:
\begin{enumerate}
\item The item $e, 2 < s(e) < 4$: the only item packed into $B$ in Step~{\rm \step{initial}}
by Observation~\ref{obs:1a}.
(Note that $e$ might end up on $A$ after renaming the bins.)
\item The item $j, 2 < s(j) < 4$, defined in Step~{\rm \step{jcheck}}.
\item The item $r, 2 < s(r) < 4$, placed into $C$ in Step~{\rm \step{qstep}} by Observation~\ref{obs:c}; $r$ is the only item in $C$ until Step~{\rm \step{b1}}.
\item The item $x$ which terminated Step~{\rm \step{qstep}}.
\end{enumerate}

\end{dfn}

There are four such items and only three bins, meaning that in the
offline optimum layout with capacity $16$, two of them are packed in
the same bin.  We will therefore argue about every possible pair,
proving that each pair is of size more than $6.8$.

Our main tool in proving the mentioned lower bounds are the
inequalities that must be true during various stages of algorithm
\tbalg, since a good situation was not reached. We now list
all the major inequalities that we will use.

\iffalse
\begin{align}
\label{eq:0}
s(A_{\before j}) &> 3 
& \mbox{Observation \ref{obs:1a}}\\
\label{eq:1}
s(B_{\before j}) + s(j) &> 6
& \mbox{Beginning of Step~\step{qstep}}\\
\label{eq:2}
s(A_{\before i}) + s(i) &> 6 
& \mbox{\gs2 not reached in Step~\step{qstep}}\\
\label{eq:3}
s(B_{\before r})+ s(r) &> 9 + \frac{s(A_{\before r})}{2} 
& \mbox{$r$ does not fit into $B|_q$}\\
\label{eq:4}
\left(s(B_{\before r}) + s(r) \right) + s(A_{\before r}) &< 15 
& \mbox{No \gs4 if $r$ packed into $B$}\\
\label{eq:5}
\left(s(B_{\before x}) + s(x) \right) + s(r) &< 15 + \frac{s(A_{\before x})}{2} 
& \mbox{No \gs4 when $x$ packed into $B|_{22}$}\\
\label{eq:6}
\left(s(B_{\before x})+ s(x) \right) + s(A_{\before x}) &< 15 + \frac{s(r)}{2}
& \mbox{No \gs4 when $x$ packed into $B|_{22}$}\\
\label{eq:7}
9 + \frac{s(A_{\before x}) + s(r)}{2} & < s(B_{\before x}) + s(x) 
 & \mbox{$x$ does not fit into $B|_{q}$}\\
\label{eq:8}
s(B_{\before x}) + (4 - s(r))  & > s(A_{\before x}) + s(x) 
& \mbox{No \gs6 if $x$ packed into $A$}\\
\label{eq:8b}
s(A_{\before x}) + s(x) + \big(4 - s(r)\big) &> s(B_{\before x}) 
& \mbox{No \gs6 if $x$ packed into $A$}\\
\label{eq:9b}
s(B_{\before x}) + \big(4 - s(A_{\before x})\big)  & > s(r) + s(x) 
& \mbox{No \gs6 if $x$ packed into $C$}\\
\label{eq:9}
s(r) + s(x) + (4 - s(A_{\before x})) & > s(B_{\before x})
& \mbox{No \gs6 if $x$ packed into $C$}
\end{align}
\fi

% We modify the display skip for more compact list
% of equations.
\newdimen\ads
\newdimen\bds
\ads=\abovedisplayskip
\bds=\belowdisplayskip
\setlength{\abovedisplayskip}{3pt}
\setlength{\belowdisplayskip}{3pt}

\begin{itemize}
\item Observation \ref{obs:1a}:
\begin{equation}\label{eq:0}
s(A_{\before j}) > 3.
\end{equation}

\item Beginning of Step~\step{qstep}:
\begin{equation}\label{eq:1}
s(B_{\before j}) + s(j) > 6.
\end{equation}

\item \gs2 not reached in Step~\step{qstep}:
\begin{equation}\label{eq:2}
s(A_{\before i}) + s(i) > 6.
\end{equation}

\item $r$ does not fit into $B|_q$:
\begin{equation}\label{eq:3}
s(B_{\before r})+ s(r) > 9 + \frac{s(A_{\before r})}{2}.
\end{equation}

\item No \gs4 if $r$ packed into $B$:

\begin{equation}\label{eq:4}
\left(s(B_{\before r}) + s(r) \right) + s(A_{\before r}) < 15.
\end{equation}

\item No \gs4 when $x$ packed into $B|_{22}$:
% \item When $x$ got packed into $B|_{22}$ in Step~\step{b1}, it did not
% cause \gs4 when summing $B$ with $C$:

\begin{equation}\label{eq:5}
\left(s(B_{\before x}) + s(x) \right) + s(r) < 15 + \frac{s(A_{\before x})}{2}. 
\end{equation}

\item No \gs4 when $x$ packed into $B|_{22}$:
% \item The item $x$ also did not cause \gs4 when summing $B$ with $A$:

\begin{equation}\label{eq:6}
\left(s(B_{\before x})+ s(x) \right) + s(A_{\before x}) < 15 + \frac{s(r)}{2}. 
\end{equation}

\item $x$ does not fit into $B|_{q}$:
\begin{equation}\label{eq:7}
s(B_{\before x}) + s(x) > 9 + \frac{s(A_{\before x}) + s(r)}{2}.
\end{equation}

\item No \gs6 if $x$ packed into $A$:
% \item The item $x$ cannot start \gs6 when packed into $A$. Comparing bin $A$ to $B$, we get:
\begin{equation}\label{eq:8}
s(A_{\before x}) + s(x) < s(B_{\before x}) + (4 - s(r)).
\end{equation}

\item No \gs6 if $x$ packed into $A$:
% We get a similar but slightly different inequality when comparing $B$ to $A$ instead:
\begin{equation}\label{eq:8b}
s(B_{\before x}) < s(A_{\before x}) + s(x) + \big(4 - s(r)\big).
\end{equation}

\item No \gs6 if $x$ packed into $C$:
% \item \gs6 could not be reached when the algorithm
% considered packing $x$ into bin $C$, comparing the bin $C$
% to $B$:
\begin{equation}\label{eq:9b}
s(r) + s(x) < s(B_{\before x}) + \big(4 - s(A_{\before x})\big).
\end{equation}

\item No \gs6 if $x$ packed into $C$:
% Again as in inequality \eq{8b} we can compare bin $B$ to $C$ and get:
\begin{equation}\label{eq:9}
s(B_{\before x}) < s(r) + s(x) + (4 - s(A_{\before x})). 
\end{equation}
\end{itemize}

% Reverting to original display skip values.
\setlength{\abovedisplayskip}{\ads}
\setlength{\belowdisplayskip}{\bds}
\goodbreak

We first show the claim using infeasible linear programming (LP)
instances formed by the above inequalities.  The specific instances
can be found in Section~\ref{app:LP} and online at \url{http://github.com/bohm/binstretch/}.

We write our LPs in the GNU MathProg language, thus they can be
verified using GNU Linear Programming Kit\footnote{One can also use
online solver at \url{https://www3.nd.edu/~jeff/mathprog/}.}.  In
Section~\ref{app:proofOfClaim} we provide formal proofs of these
lemmas for completeness.

In our LPs we use a variable \texttt{i} for $s(i)$, the size of item $i$,
and \texttt{X} for $s(X)$ where $X$ is a bin. Instead of $s(X_{\before i})$
we write \texttt{X\_i}.

Since our inequalities are mostly strict and LPs cannot
contain strict inequalities, we add a non-negative variable \texttt{eps} (epsilon)
which allows us to turn strict inequalities to non-strict.
More precisely, we change an inequality of type \texttt{A < B} into \texttt{A + eps <= B}
and we maximize the value of \texttt{eps}; we can do this, because our LPs
do not need another objective function. If the optimal value of \texttt{eps} 
is zero, or if the LP is infeasible, then also the original system or strict inequalities
is infeasible as well. Otherwise, if the optimal value of \texttt{eps} is positive,
then all the strict inequalities can be satisfied.

Our first lemma establishes that $j$ is actually the only item that
is packed into $B$ during Step~\step{qstep}, which intuitively means that
$j$ is not too small:

\begin{lem}\label{lem:onej}
Assume that no good situation is reached until Step~{\rm \step{b1}}. Then
it holds that during Step~{\rm \step{qstep}}, only $j$ is packed into $B$.
\end{lem}

\begin{proof}
We first prove that no two additional items $j_2, j_3$ can be packed into $B$
during Step~\step{qstep}. Assuming the contrary, we get $s(B_{\before j_2}) +
s(j_2) + s(j_3) > 6 + 2 + 2 = 10$. With that load on $B$, we consider
the packing at the end of Step~\step{qstep}, when the item $x$ arrived. If
$s(x) + s(C_{\before x}) < 9$, we get \gs6 by placing $x$ into $C$ since $s(A)>3$, so it must be true
that $s(x) + s(C_{\before x}) > 9$, which means $s(x) >
5$. This is enough for us to place $x$ into $B|_{22}$
(where it fits, otherwise we are in \gs7) and reach \gs3.

This contradiction gives us that at most one additional item $j_2$ can
be packed into $B$ during Step \step{qstep}. We will now prove that
even $j_2$ does not exist, again by contradiction.

We split the analysis into two cases depending on which of $j_2$ and $r$ arrives first.

\smallskip
\noindent \textbf{Case 1.} The item $r$ is packed
before $j_2$, meaning $s(B_{\before x}) = s(B_{\before r}) + s(j_2)$.
We create a linear program from inequalities \eq{0}, \eq{3}, \eq{6}, \eq{8b} 
and $s(A_{\before r}) \ge s(A_{\before j})$ (since $r$ arrives after $j$).
We also add $s(r)<4$ and $s(j_2) > 2$, since
$s(A_{\before j_2}) < 4$ and $j_2$ did not fit into $A|_{6}$.
We obtain LP1 for which the optimal value is 0, a contradiction.

\smallskip

\noindent \textbf{Case 2.} In the remaining case, $j_2$ arrives before $r$.
We create an LP from \eq{1}, \eq{2}, \eq{4}, \eq{6}, \eq{9}, $s(r) < 4$,
$s(A_{\before j_2}) \le s(A_{\before r}) \le s(A_{\before x})$,
and $s(B_{\before x}) = s(B_{\before r}) = s(B_{\before j}) + s(j) + s(j_2)$
(the last two are only true here in Case 2, where $r$ arrived later than $j_2$).
The resulting LP2 is infeasible.
\end{proof}

Having established that only one item $j$ is packed into $B$ during
Step \step{qstep}, we can start deriving lower bounds on pairs of
items from the set $\{e,j,r,x\}$. We will prove these bounds similarly
to Lemma \ref{lem:onej} by infeasible LPs from bounds that arise from
evading various good situations.

\begin{lem}\label{lem:eplusr}
Suppose that $e$ and $r$ are items as described in Definition \ref{dfn:fouritems} and suppose
also that no good situation was reached during Step~{\rm \step{qstep}} of the algorithm \tbalg.
Then, $s(e) + s(r) \ge  s(B_{\before j}) + s(r) > 6.8$.
\end{lem}

\begin{proof}
First of all, it is important to note that the item $e$ may be
packed on $A$ or on $B$.
% Regardless, if we prove $s(B_{\before j}) + s(r) >
% 6.8$, we will be done: $j$ is placed on top of $B_{\before j}$
% and we have $s(B_{\before j}) \le s(A_{\before j})$ by Step~\step{rename}.
Since either $B_{\before j}$, or $A_{\before j}$ contains solely $e$
by Observation~\ref{obs:1a}, we get that either $s(B_{\before j}) = s(e)$,
or $s(B_{\before j}) \le s(A_{\before j}) = s(e)$. Thus it is sufficient to prove
$s(B_{\before j}) + s(r) > 6.8$.

We use \eq{3}, \eq{7}, \eq{8}, $s(j) < 4$, \[s(B_{\before j}) \le s(A_{\before j}) \le s(A_{\before r}) \le s(A_{\before x}),\]
and a converse of the claim and obtain LP3 with the optimal value equal to 0.
\end{proof}

\begin{lem}\label{lem:eplusj}
Suppose that $e$ and $j$ are items as described in Definition \ref{dfn:fouritems} and suppose
also that no good situation was reached by the algorithm \tbalg.
Then, $s(e) + s(j) \ge  s(B_{\before j}) + s(j) > 7.6$.
\end{lem}

\begin{proof}
The same argument as in Lemma \ref{lem:eplusr} gives us $s(e) + s(j) \ge  s(B_{\before j}) + s(j)$.
We therefore aim to prove $s(B_{\before j}) + s(j) > 7.6$. 
We create LP4 from \eq{7}, \eq{9b}, $s(B_{\before j}) + s(r) > 6.8$ by Lemma \ref{lem:eplusr},
and $s(B_{\before j} \leq s(A_{\before x})$ for which 0 is the optimum again.

\end{proof}

\begin{lem}\label{lem:rplusj}
Suppose that $j$ and $r$ are items as described in Definition \ref{dfn:fouritems} and suppose
also that no good situation was reached by the algorithm \tbalg. Then, $s(r) + s(j) > 7$.
\end{lem}

\begin{proof} % PV: this seems to be simple enough to have the old proof
Starting with \eq{3}:
\[ s(B_{\before j}) + s(j) + s(r) > 9 + \frac{s(A_{\before r})}{2}  \]
and using $s(B_{\before j})\le s(A_{\before j})\le s(A_{\before r})$ with $s(B_{\before j}) < 4$, we have:
\begin{align*}
s(j) + s(r) &> 9 + \left( \frac{s(A_{\before r})}{2} - s(B_{\before j}) \right) \\
            & \ge  9 - \frac{s(B_{\before j})}{2} > 7. \qquad 
\end{align*}
\end{proof}

\begin{lem}\label{lem:xplus}
Suppose that $x,e,j,r$ are items as described in Definition \ref{dfn:fouritems}.
Suppose also that no good situation was reached by the algorithm \tbalg.
Then,
\[ s(x) > 4 \quad \text{and} \quad s(x) + \min(s(j),s(e),s(r)) > 6.8. \]
\end{lem}

\begin{proof}
With all the previous lemmas in place, the proof is simple enough. We first
observe that $s(x) > 4$; this is true because $s(B_{\before j}) + s(j) < 4+4 = 8$ and
$s(B_{\before x}) + s(x) > q \geq 12$.

Since the sizes of the remaining three items $\{e,r,j\}$ are bounded from above
by $4$ but their pairwise sums are always at least $6.8$, we have
that $\min\{e,r,j\} > 2.8$, which along with $s(x) > 4$ gives us the
required bound.
\end{proof}

From Lemmas \ref{lem:eplusr}, \ref{lem:eplusj}, \ref{lem:rplusj} and
\ref{lem:xplus} we get a portion of Claim \ref{clm:nobigitems}: if
three big items $b_1, b_2, b_3$ exist in the offline layout, then one of these
items needs to be packed together with at least two items from the set
$\{e,j,r,x\}$, and therefore $\min(s(b_1),s(b_2),s(b_3)) < 9.2$. The
second bound $s(r) > 2.8$ follows from Lemma \ref{lem:eplusr} and the
fact that $s(e) < 4$.

All that remains is to prove the bound on $s(B)$, which we do in the following lemma:

\begin{lem}\label{lem:boundonB}
Suppose that no good situation was reached in the algorithm \tbalg during Step~{\rm \step{qstep}}.
Then, after pla\-cing $x$ into $B$ in Step~{\rm \step{b1}}, it holds that $s(B) < 12.8$.
\end{lem}

\begin{proof}

As before, we will use our inequalities to derive an LP showing the desired
bound. As we have argued above, Lemma \ref{lem:eplusr} gives us that
$s(r) > 2.8$. We also use inequalities \eq{6}, \eq{9b},
$s(B_{\before j}) \le s(A_{\before j}) \le s(A_{\before x})$
(this is true because we reorder the bins $B$, $A$ in Step~\step{rename}),
and $s(j) < 4$ to get LP5 with 0 being the optimal value.
\end{proof}

With Lemma \ref{lem:boundonB} proven, we have finished the proof of Claim \ref{clm:nobigitems}
and completed the analysis of the algorithm \tbalg.

