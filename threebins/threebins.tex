\chapter{An Algorithm Fine-Tuned for Three Bins}
\label{chap:threebins}
% \addcontentsline{toc}{chapter}{Algorithms Fine-Tuned for Three Bins}

In this section, we focus on instances with exactly three bins. 
We prove the following:

\begin{thm}\label{thm:3binsalgo}
There exists an algorithm that solves
\binstretch for three bins with stretching factor $1+3/8=1.375$.
\end{thm}

Our final online algorithm uses several subroutines, one of which is
the classical online algorithm \FF:

\smallskip
\algobox{
\textbf{Subroutine \FF:}
\begin{compactenum}[(1)]
\item Set an ordering of your bins. 
\item For every incoming item $i$:
\item \indentskip Pack $i$ into the first bin where $i$ fits below or to the limit $t$.
\item \indentskip If no such bin exists, report failure.
\end{compactenum}
}
\smallskip

The three bins of our setting are named $A$, $B$, and $C$.  We
exchange the names of bins sometimes during the course of the
algorithm.

Throughout the proof, we will need to argue about loads of the bins
$A$, $B$, $C$ before various items arrived. The following notation
will help us in this endeavour:

Suppose that $A$ is a bin and $x$ is an item that gets packed at some
point of the algorithm (not necessarily into $A$).  Then $A_{\before
x}$ will indicate the set of items that are packed into $A$ just
before $x$ arrived.

We scale the input sizes by $16$. The stretched bins in our setting
therefore have capacity $22$ and the optimal offline algorithm can
pack all items into three bins of capacity $16$ each.

\section{Algorithm overview}\label{sec:3:algoverview}

If we were to design a new algorithm from scratch, we would probably
start with trying to pack first all items in a single bin, as long as
possible. In general, this is the strategy that the final algorithm
will also follow. However, somewhat surprisingly, it turns out that
from the very beginning we need to put items in two bins even if the
items as well as their total size are relatively small.

It is clear that we have to be very cautious about exceeding a load of
6. For instance, if we put 7 items of size 1 in bin $A$, and 7 such items in $B$,
then if two items of size 16 arrive, the algorithm will have a load of 
at least 23 in some bin. Similarly, we cannot assign too much to a single bin:
putting 20 items of size 0.5 all in bin $A$ gives a load of 22.5 somewhere if 
three items of size 12.5 arrive next. (Starting with items of size 0.5
guarantees that there is an optimal solution with bins of capacity 16.) 

On the other hand, it is useful to keep one bin empty for some time; many
problematic instances end with three large items such that one of them has
to be placed in a bin that already has high load. Keeping one bin free ensures
that such items must have size more than 11 (on average), which limits the
adversary's options, since all items must still fit into bins of size 16.

Deciding when exactly to start using the third bin and when to cross the
threshold of 6 for the first time was the biggest challenge in designing
this algorithm: both of these events should preferably be postponed as long
as possible, but obviously they come into conflict at some point.

