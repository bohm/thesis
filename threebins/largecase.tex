\subsection{The large case}
We now assume that $s(j) > 6$. Our goal in both the large case and the
standard case will be to show that in the near future either a good
situation is reached or several large items arrive, but \tbalg is able
to pack them nonetheless.

Let us start by recalling the relevant steps of the algorithm:

\smallskip
\algobox{
\step{large} Set $p \defeq 6 + s(j)$; apply \GSFF{$A|_p, B|_{4}$}.\\
\step{wcheck} If the next item $w$ fits into $A|_{22}$:\\
\step{wfits}  \indentskip \GSFF{$A|_{22}, B|_{22}, C|_{22}$}.\\
\step{otherwise} Else: \\
\step{wwontfit}  \indentskip \GSFF{$A|_p, B|_{22}, C|_{22}$}
%, where $p \defeq 6 + s(j)$.
}

By choosing the limit $p$ to be $s(j)+6$ in Step \step{large}, we make
enough room for $j$ to be packed into $A$. We also ensure that
any item $i$ larger than $6$ that cannot be placed into $A$ with
capacity 22 must satisfy $s(i) + s(j) > 16$ and so $i$ cannot be with
$j$ in the same bin in the offline optimum packing.

Let us define $A_S$ as the set of items in $A$ of size less than $6$
(packed before or after $j$).  We note the following:

\begin{obs}\label{obs:nothingonc}

\leavevmode
\begin{enumerate}
\item During Step {\rm \step{large}}, if $B$ contains any item, it is true that
$s(A_S) + s(B) > 6$.
\item If no good situation is reached, the item $w$ ending Step {\rm \step{large}} satisfies $s(w) > 6$.
\end{enumerate}

\end{obs}

\begin{proof}
The first point follows immediately from our choice of $p$ and \GSFF{$A|_p, B|_4$}.

For the second part of the observation, consider the item $w$ that
ends Step~\step{large} and assume $s(w) \le 6$. The possibility that
$s(w) \in  [4,6]$ is excluded due to \gs2. The case $s(B_{\before w}) \ge  3$ is
also excluded, as this would imply \gs5 with $j$ in $A$.

Since $s(B_{\before w}) + s(w) > 6$, the only remaining possibility is $s(B_{\before w}) \in  [2,3), s(w) \in 
(3,4)$. Even though $w$ does not fit into $A|_p$, if we were to pack
$w$ into $A|_{22}$, we can use the first point of this observation and get
$s(A) + s(B) \ge  s(j) + \big(s(A_S) + s(B_{\before w})\big) + s(w) > 6 + 6 + 3 = 15$, enough for \gs4 as $s(C) = 0$. The algorithm
\GSFF{$A|_{p}, B|_{4}$} in Step \step{large} will notice this
possibility and will pack $w$ into $A$, where it will always fit, as
$s(A_{\before w}) < 15$ by \gs4.
\end{proof}

We now split the analysis based on which branch is entered in Step \step{wcheck}:

\smallskip
\noindent
{\bf Case 1:} Item $w$ fits into bin $A$; we enter Step \step{wfits}.

We first note that $s(A)+s(B)<15$, else we are in \gs4 since $C$ is still empty. This inequality also
implies $s(B) =~0$, otherwise we have
\[s(A) + s(B) = s(w) + s(j) + (s(A_S) +s(B)) > 18\]
via Observation~\ref{obs:nothingonc} and this is enough for \gs4.

We continue with Step~\step{wfits} until we reach a good situation or
the end of input.  Suppose three items $x,y,z$ arrive such that none of them 
can be packed into $A$ and we do not reach a good situation. We will prove that
this cannot happen. We make several quick observations about those items:

\begin{enumerate}
\item We have $s(x) > 7$ because $s(A_{\before x}) < 15$ or we reach \gs4. The item $x$ is packed into $B$.
\item At any point, $B$ contains at most one item, otherwise $s(A) + s(B) > 22 + 7 > 26$, reaching \gs1.
\item We have $s(y) > 9$ because $\min(s(A_{\before y}), s(B_{\before y})) < 13$ by \gs1. The item $y$ is packed into $C$.
\item The bin $C$ contains also at most one item, similarly to $B$.
\item Again, we have $s(z) > 9$ similarly to $y$. The item $z$ does not fit into any bin.
\end{enumerate}

From our observations above, we get $s(x) + s(y) > 22$, $s(x) + s(z) >
22$, $s(y) + s(z) > 22$. Therefore, at least two of the items
$\{x,y,z\}$ are of size at least $11$ and the third one is larger than 6.
However, both items $j$ and $w$ have size at least $6$,
and there is no way to pack $j,w,x,y,z$ into three bins of capacity $16$, a contradiction.

\smallskip
\noindent \textbf{Case 2:} Item $w$ does not fit into bin $A|_{22}$. The choice
of $p$ gives us $s(j)+s(w)>16$. Item $w$ is placed on $B$.

The limit $p$ gives us an upper bound on the volume of small items
$A_S$ in $A$, namely $s(A_S) \le 6$.  An easy argument gives us a similar bound on $B$,
namely if $B_S \defeq B \setminus \{w\}$, then $s(B_S) < 4$. Indeed, we have
$26 > s(A) + s(B) > 22 + s(B_S)$, the first inequality implied by not
reaching \gs1.

In Case 2, it is sufficient to consider two items $x,y$ that do not
fit into $A|_{p}$ or $B|_{22}$. We have:

\begin{enumerate}
\item Using $s(B_S)<4$, we have $s(x) + s(w) > 18$ and $s(y) + s(w) > 18$.
\item None of the items $x,y$ fits into $A|_{22}$. If say $x$ did fit, then we
use the fact that $x$ does not fit into $B|_{22}$ and get
$s(B) + s(A) = \big(s(B_{\before x}) + s(x)\big) + s(A_{\before x}) > 22 + s(j) > 26$ and we reach \gs1.
\item The choice of the limit $p$ on $s(A)$ implies $s(x) + s(j) > 16$ and $s(y) + s(j) > 16$.
\item Since $\min(s(A),s(B)) < 13$ at all times by \gs1, we have $s(x)>9$ and $s(y) > 9$.
\item The items $x$ and $y$ do not fit together into $C$, or we would have
$s(C) + s(A) > 22 + s(y) > 26$.  This implies $s(x) + s(y) > 22$.

\end{enumerate}

From the previous list of inequalities and using $s(j) + s(w) > 16$,
we learn that no two items from the set $\{j,w,x,y\}$ can be together
in a bin of size $16$. Again, this is a contradiction with the
assumptions of \binstretch.