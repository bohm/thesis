\section{Good situations}\label{sec:3:gs}
Before stating the algorithm itself, we list a number of \textit{good
situations} (GS). These are configurations of the three bins which
allow us to complete the packing regardless of the following input.

It is clear that the identities of the bins are not important here;
for instance, in the first good situation, all we need is that
\emph{any} two bins together have items of size at least 26. We have
used names only for clarity of presentation and of the proofs.

\begin{dfn}
A \textit{partial packing} of an input sequence $S$
is a function $p: S_1 \to \{A,B,C\}$ that assigns a bin to each
item from a prefix $S_1$ of the input sequence $S$.
\end{dfn}


\begin{goodsit}\label{lem:gs1}
Given a partial packing such that $s(A) + s(B) \geq 26$, 
%and $s(C)$ is arbitrary, 
there exists an online algorithm that can finish
the packing with capacity $22$.
\end{goodsit}

\begin{proof}
Since the optimum can pack into three bins of size $16$, the total
size of items in the instance is at most $3\cdot 16=48$. If two bins
have size $s(A) + s(B) \geq 26$, all the remaining items (including
the ones already placed on $C$) have size at most $22$.  Thus we can
pack them all into bin $C$.
\qed
\end{proof}

\begin{goodsit}\label{lem:gs2}
Given a partial packing such that $s(A) \in [4, 6]$, 
%and $s(B)$ and $s(C)$ are arbitrary, 
there exists an online algorithm that can finish
the packing with capacity $22$.
\end{goodsit}

\begin{proof}
Let $A$ be the bin with size between $4$ and $6$ and $B$ be one of the
other bins (choose arbitrarily). Put all the items greedily into
$B$. When an item $x$ does not fit, put it into $A$, where it fits, as
 $s(A_{\before x})\le 6$. Now $s(B_{\before x})+s(x)>22$. In addition, $s(A_{\before x})\ge 4$ by the assumption.  Together we
have $s(A_{\before x}) + s(B_{\before x})+s(x) \geq 26$, which is \gs1.
\qed
\end{proof}

From now on, we assume that each bin $X\in\{A,B,C\}$ satisfies $s(X) \not\in [4,6]$, otherwise we are in \gs2.

\begin{goodsit}\label{lem:gs3}
Given a partial packing such that $s(A) \ge 15$ and either
{\rm(i)} $s(B)+s(C)\ge 22$ or
{\rm(ii)} $s(C)<4$ and $s(B)$ is arbitrary,
there exists an online algorithm that can finish the packing with capacity $22$.
\end{goodsit}

\begin{proof}
(i) We have
%If $s(B)+s(C)\ge22$, then 
$\max(s(B),s(C))\ge11$, so 
we are in \gs1 on bins
$A$ and $B$ or on bins $A$ and $C$. %Else, if $s(C)<4$, w

(ii) We pack arriving items into $B$. If $s(B) \geq
11$ at any time, we apply \gs1 on bins $A$ and $B$. Thus we can
assume $s(B) < 11$ and we cannot continue packing into $B$ any
further. This implies that an item $i$ arrives such that $s(i) >
11$. As $s(C_{\before i})<4$, we pack $i$ into it and apply \gs1 on
bins $A$ and $C$.\qed
\end{proof}

\begin{goodsit}\label{lem:gs4}
Given a partial packing such that 
$s(A) +s(B) \geq 15+\frac{1}{2}s(C)$, $s(B) < 4$, and $s(C) < 4$,
there exists an online algorithm that can finish the packing with capacity $22$.
\end{goodsit}

\begin{proof}
Let $c$ be the value of $s(C)$ when the conditions of this good situation
hold for the first time.
We run the following algorithm until we reach \gs1 or \gs3:

\smallskip
\algobox{
\begin{compactenum}[(1)]
\item \label{gs4:step1} If the incoming item $i$ has $s(i)\ge 11-\frac{1}{2}c$, pack $i$ into~$B$.
\item Else, if $i$ fits on $A$, pack it there.
\item Otherwise pack $i$ into $C$.
\end{compactenum}
}
\medskip

If at any time an item is to be packed into $B$ by Step~\ref{gs4:step1} (it always fits since we maintain $s(B) < 4$),
then $s(A)+s(B)\ge 26$ and we reach \gs1. In the event that no item
is packed into $B$, we reach \gs3 (with $B$ in the role of $C$) whenever the algorithm brings the size
of $A$ to or above 15.

The only remaining case is when $s(A)<15$ throughout the algorithm and
several items with size in the interval $I \defeq
(22-s(A),11-\frac{1}{2}c)$ arrive. These items are packed into $C$.
Note that $I \subseteq (7,11)$ and that the lower bound of $I$ may decrease
during the course of the algorithm.

The first two items with size in $I$ will fit together,
since $2(11-\frac{1}{2}c) + c = 22$. With two such items packed into $C$, we know that the load $s(A) + s(C)$ is at least $s(A) + 2(22-s(A)) = 44 - s(A) > 29$ and we have reached \gs1, finishing the analysis. \qed

\end{proof}

\begin{goodsit}\label{lem:gs5}
Given a partial packing such that a new item $a$ with $s(a)>6$ is packed
into bin $A$, $s(B_{\before a}) \in [3,4)$, and $s(C_{\before a})=0$,
there exists an online algorithm that can finish the packing with capacity $22$.
\end{goodsit}

\begin{proof}
%We have $s(B) \in [3,4)$, otherwise we reach \gs2.
Pack all incoming items into $A$ as long as it is possible. 
If at some point $s(A) \geq 12$, we are in \gs4, and so we assume the contrary.
Therefore, $s(A) < 12$ and an item $i$ arrives which cannot be packed into $A$. 

Place $i$ into $B$. If $s(i) \geq 12$, we apply \gs3. We thus have
$s(i) \in (10,12)$ and $s(A_{\before i}) > 22 - s(i) > 10$. Continue with \FF
on bins $B$, $A$, and $C$ in this order. (That is, pack an incoming item into the first bin $X$
in which the item fits. If there is no such bin, stop.)

We claim that \gs1 is reached at the latest after \FF has
packed two items, $x$ and $y$, on bins other than $B$. If one of them (say $x$) is
packed into bin $A$, this holds because $s(x) + s(B_{\before x})>22$
and $s(A_{\before x})>10$---enough for \gs1.
If both items do not fit in $A$, they are both larger than 10, since $s(A_{\before i})<12$ and nothing gets packed into $A$ after item $i$.
We will show by contradiction that this cannot happen.

As $s(A_{\before x}) < 12$ from our previous analysis, we note that $s(x), s(y) > 10$.
We therefore have three items $i,x,y$ with $s(i),s(x),s(y)>10$ and
an item $s(a) > 6$ from our initial conditions. These four items
cannot be packed together by any offline algorithm into three bins of
capacity 16, and so we have a contradiction with $s(x), s(y) > 10$.
\qed
\end{proof}

\begin{goodsit}\label{lem:gs6} 
If $s(C)<4$, $s(B)>6$ and $s(A)\ge s(B)+ 4-s(C)$, 
there exists an online algorithm that can finish the packing with capacity $22$.
\end{goodsit}

\begin{proof}
Pack all items into $A$, until an item $x$ does not
fit.  At this point $s(A_{\before x})+s(x)>22$.  If $x$ fits on $B$, we put it there
and reach \gs1 because $s(B_{\before x})>6$. Otherwise, $x$ definitely fits on $C$
because $s(C_{\before x})<4$ by assumption.  By the condition on $s(A)$, we have % $s(A)+s(C)-4+x \ge s(B)+x > 22$, so 
$s(x) + s(A_{\before x})+ s(C) \ge s(x) + s(B)+4>26$, and we are in \gs1 again.
\qed
\end{proof}

%\begin{cor}\label{cor:gs6}
%Suppose 
%$s(C)\le s(B)< 6< s(A)$. 
%If an item of size at least $s(A)-s(C)+4-s(B)$ arrives, we reach \gs6
%by placing it on $B$ or $C$.
%If an item of size at most $s(A)-s(C)-(4-s(B))$ arrives, it can be placed on $C$; we reach a good situation if $s(C)\ge4$ as a result.
%\end{cor}
%
%\begin{proof}
%A large enough item can be assigned to $C$ (or $B$) and will bring us to the situation of the
%last lemma. A small enough item can be placed on $C$.
%\qed
%\end{proof}

\begin{goodsit}\label{lem:gs7}
Consider the arrival of an item $x$.
If it holds that
\begin{itemize}
\item $s(A_{\before x})<4$, 
\item $s(C_{\before x})<4$, % and $s(B)>6$. If 
\item $s(B_{\before x})\le 9+\frac{1}{2}(s(A_{\before x})+s(C_{\before x}))$, %and for a new item $x$ we have
\item and $s(B_{\before x})+s(x)>22$,
\end{itemize}
then there exists an online algorithm that packs all remaining items into
three bins of capacity $22$.
\end{goodsit}

\begin{proof}
We have $s(B_{\before x}) > 22 - s(x) > 6$ and 
\begin{align*}
s(x) > 22-s(B_{\before x}) &\ge 13 - \frac{1}{2}(s(A_{\before x})+s(C_{\before x}))\\ &\ge s(B_{\before x})+4-s(A_{\before x})-s(C_{\before x}).
\end{align*}

Placing $x$ on $A$ we increase $s(A)$ to at least $s(B_{\before x})+4-s(C_{\before x})$
and we reach \gs6. % by Corollary~\ref{cor:gs6}.
\qed
\end{proof}


\subsection{Good Situation First Fit}\label{sec:3alg}

Throughout our algorithm, we often use a special variant of \FF\/
which tries to reach good situations whenever possible. This
variant can be described as follows:

\begin{dfn}\label{dfn:gsff}
Let $\calL=(X|_k, Y|_l, \ldots)$ denote a list of bins $X,Y, \ldots $ where each
bin $X$ has an associated capacity $k$ satisfying $s(X)\leq k$.
% originally the limit $k$ was integral but it doesn't have to be
\GSFF{$\calL$} (Good Situation First Fit) is an online algorithm for bin stretching that works as follows:

\vspace{1.5ex}
\algobox{ %TODO PV this box might be written a nicer way
\textbf{Subroutine \GSFF{$\calL$}:}
For each item $i$:\\
If it is possible to pack $i$ into any bin (including bins not in $\calL$, and
using capacities of 22 for all bins)
so that a~good situation is reached,
do so and continue with the algorithm of the %\\
relevant good situation.\\
Otherwise, traverse the list $\calL$ in order and pack $i$ into the first bin $X$
such that $X|_k \in \calL$ and $s(X) + s(i) \leq k$.
If there is no such bin, stop.
}
\end{dfn}

For example, \GSFF{$A|_4, B|_{22}$} checks whether
either the packing $(A\cup \{i\},B,C)$, $(A,B\cup \{i\},C)$ or $(A,B,C\cup \{i\})$ is a partial packing of any good
situation. If this is not the case, the algorithm packs $i$ into bin
$A$ provided that $s(A)+s(i)\leq 4$. If $s(A) + s(i)>4$, the algorithm
packs $i$ into bin $B$ with capacity $22$. If $i$ cannot be placed
into $B$, \GSFF{$A|_4, B|_{22}$} halts and another
online algorithm must be applied to pack $i$ and subsequent items.