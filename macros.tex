%%% This file contains definitions of various useful macros and environments %%%
%%% Please add more macros here instead of cluttering other files with them. %%%

%%% Minor tweaks of style

% These macros employ a little dirty trick to convince LaTeX to typeset
% chapter headings sanely, without lots of empty space above them.
% Feel free to ignore.
\makeatletter
\def\@makechapterhead#1{
  {\parindent \z@ \raggedright \normalfont
   \Huge\bfseries \thechapter. #1
   \par\nobreak
   \vskip 20\p@
}}
\def\@makeschapterhead#1{
  {\parindent \z@ \raggedright \normalfont
   \Huge\bfseries #1
   \par\nobreak
   \vskip 20\p@
}}
\makeatother

% This macro defines a chapter, which is not numbered, but is included
% in the table of contents.
\def\chapwithtoc#1{
\chapter*{#1}
\addcontentsline{toc}{chapter}{#1}
}

% Draw black "slugs" whenever a line overflows, so that we can spot it easily.
\overfullrule=1mm

%%% Macros for definitions, theorems, claims, examples, ... (requires amsthm package)

\theoremstyle{plain}
\newtheorem{thm}{Theorem}
\newtheorem{lem}[thm]{Lemma}
\newtheorem{clm}[thm]{Claim}
\newtheorem{obs}[thm]{Observation}
\newtheorem{goodsit}{Good Situation}
\newtheorem{gengoodsit}{Good Situation} % to avoid pdftex warning about duplicate labels

\theoremstyle{definition}
\newtheorem{dfn}{Definition}
\newtheorem{prb}{Problem}

\theoremstyle{remark}
\newtheorem*{cor}{Corollary}
\newtheorem*{rem}{Remark}
\newtheorem*{example}{Example}

%%% An environment for proofs

%%% FIXME %%% \newenvironment{proof}{
%%% FIXME %%%   \par\medskip\noindent
%%% FIXME %%%   \textit{Proof}.
%%% FIXME %%% }{
%%% FIXME %%% \newline
%%% FIXME %%% \rightline{$\square$}  % or \SquareCastShadowBottomRight from bbding package
%%% FIXME %%% }

%%% An environment for typesetting of program code and input/output
%%% of programs. (Requires the fancyvrb package -- fancy verbatim.)

\DefineVerbatimEnvironment{code}{Verbatim}{fontsize=\small, frame=single}

%%% The field of all real and natural numbers
\newcommand{\R}{\mathbb{R}}
\newcommand{\N}{\mathbb{N}}

%%% Useful operators for statistics and probability
\DeclareMathOperator{\pr}{\textsf{P}}
\DeclareMathOperator{\E}{\textsf{E}\,}
\DeclareMathOperator{\var}{\textrm{var}}
\DeclareMathOperator{\sd}{\textrm{sd}}

%%% Transposition of a vector/matrix
\newcommand{\T}[1]{#1^\top}

%%% Various math goodies
\newcommand{\goto}{\rightarrow}
\newcommand{\gotop}{\stackrel{P}{\longrightarrow}}
\newcommand{\maon}[1]{o(n^{#1})}
\newcommand{\abs}[1]{\left|{#1}\right|}
\newcommand{\dint}{\int_0^\tau\!\!\int_0^\tau}
\newcommand{\isqr}[1]{\frac{1}{\sqrt{#1}}}

%%% Various table goodies
\newcommand{\pulrad}[1]{\raisebox{1.5ex}[0pt]{#1}}
\newcommand{\mc}[1]{\multicolumn{1}{c}{#1}}


% custom macros

\newcommand\binstretch{{\sc Online Bin Stretching}\xspace}
\newcommand\binpacking{{\sc Bin Packing}\xspace}
\newcommand\scheduling{\textsc{Online Scheduling}\xspace}
\newcommand\Nat{\mathbb{N}}

\newcommand\eps\varepsilon
\newcommand\calA{{\mathcal A}}
\newcommand\Cprime{{\mathcal C'}}
\newcommand\calB{{\mathcal B}}
\newcommand\calC{{\mathcal C}}
\newcommand\calD{{\mathcal D}}
\newcommand\calE{{\mathcal E}}
\newcommand\calF{{\mathcal F}}
\newcommand\calH{{\mathcal H}}
\newcommand\calI{{\mathcal I}}
\newcommand\calL{{\mathcal L}}
\newcommand\calR{{\mathcal R}}
\newcommand\Aprime{{\mathcal A'}}
\newcommand{\hmodfour}{{h}}
\newcommand{\before}{\leftarrow}

\newcommand{\Rbar}{R_{\mbox{\rm\scriptsize first}}}
\newcommand\mycase[1]{\vspace{0.4ex}\noindent{\bf #1}}
\newcommand\mycasesp[1]{\vspace{1ex}\noindent{\bf #1}}

\def\indentskip{\hskip 1.5em}
\def\indentskiptwodigit{\hskip 1em}

\newcommand\algobox[1]{\begin{center}
\fbox{\parbox[c]{0.95\textwidth}{#1}}
\end{center}
}

\def\O{{\mathcal{O}}}
\let\csc=\textsc
\def\GSFF#1{{\rm\csc{GSFF(#1)}}}
\def\FF{{\rm\csc{First Fit}}\xspace}
\def\tbalg{\textsc{Evasive}\xspace}
\def\SP{{\textsc{Large Item Early}}\xspace}
\def\enumparams{\itemsep1pt \parskip0pt}

\def\step#1{{\ref{fp:#1}}}
\def\eq#1{{(\ref{eq:#1})}}
\def\gs#1{GS\ref{lem:gs#1}}

% box related macros
\def\frombox#1{\hbox to 1em{{\bf #1}\hss}}
\def\tobox#1{\hbox to 2.3em{{\bf #1}\hss}}
\def\arrow{\hbox to 1.5em{$\rightarrow$}}

\makeatletter
\newcommand*{\defeq}{\mathrel{\rlap{%
                     \raisebox{0.3ex}{$\m@th\cdot$}}%
                     \raisebox{-0.3ex}{$\m@th\cdot$}}%
                     =}
\makeatother


\newcommand\ALG{\mathrm{ALG}}
\newcommand\OPT{\mathrm{OPT}}
% macros for the lower bound chapter
\renewcommand\Game{{\mathrm{BSG}}}
\newcommand\algo{\textsc{Algorithm}\xspace}
\newcommand\adversary{\textsc{Adversary}\xspace}
\newcommand\evaladv{\textsc{EvaluateAdversary}\xspace}
\newcommand\evalalg{\textsc{EvaluateAlgorithm}\xspace}
\newcommand\minimax{\textsc{Minimax}\xspace}
\newcommand\Main{\textsc{Main}\xspace}
\newcommand\Sequential{\textsc{Sequential}\xspace}

\newcommand\Test{\textsc{Test}\xspace}
\newcommand\DynprogMax{\textsc{DynprogMax}\xspace}
\newcommand\Query{\textsc{Query}\xspace}
\newcommand\MaxFeas{\textsc{MaxFeas}\xspace}
\newcommand\bfd{\textsc{Best Fit Decreasing}\xspace}
\newcommand\obf{\textsc{Online Best Fit}\xspace}
\newlength{\elseskiplen}
\settowidth{\elseskiplen}{Else: }
\newcommand\elseskip{\hskip\elseskip}


\newcommand\todo[1]{\textcolor{red}{#1}}